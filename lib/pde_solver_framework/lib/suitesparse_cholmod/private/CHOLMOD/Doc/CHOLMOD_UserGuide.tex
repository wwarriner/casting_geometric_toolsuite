%-------------------------------------------------------------------------------
% The CHOLMOD/Doc/UserGuide.tex file.
%-------------------------------------------------------------------------------

\documentclass[11pt]{article}

\newcommand{\m}[1]{{\bf{#1}}}       % for matrices and vectors
\newcommand{\tr}{^{\sf T}}          % transpose
\newcommand{\new}[1]{\overline{#1}}

\topmargin 0in
\textheight 8.5in
\oddsidemargin 0pt
\evensidemargin 0pt
\textwidth 6.5in

\begin{document}

\author{Timothy A. Davis \\
DrTimothyAldenDavis@gmail.com, http://www.suitesparse.com }
\title{User Guide for CHOLMOD: a sparse Cholesky factorization and
modification package}

\date{VERSION 3.0.13, Dec 20, 2018}
\maketitle

%-------------------------------------------------------------------------------
\begin{abstract}
    CHOLMOD\footnote{CHOLMOD is short for CHOLesky MODification,
    since a key feature of the package is its ability to update/downdate
    a sparse Cholesky factorization}
    is a set of routines for factorizing sparse symmetric positive
    definite matrices of the form $\m{A}$ or $\m{AA}\tr$, updating/downdating
    a sparse Cholesky factorization, solving linear systems, updating/downdating
    the solution to the triangular system $\m{Lx}=\m{b}$, and many other sparse
    matrix functions for both symmetric and unsymmetric matrices.
    Its supernodal Cholesky factorization
    relies on LAPACK and the Level-3 BLAS, and obtains a substantial fraction
    of the peak performance of the BLAS.  Both real and complex matrices
    are supported.  
    It also includes a non-supernodal $\m{LDL}^T$ factorization method
    that can factorize symmetric indefinite matrices if all of their
    leading submatrices are well-conditioned ($\m{D}$ is diagonal).
    CHOLMOD is written in ANSI/ISO C, with both
    C and MATLAB interfaces.  This code works on Microsoft Windows and many versions
    of Unix and Linux.
\end{abstract}
%-------------------------------------------------------------------------------

CHOLMOD Copyright\copyright 2005-2015 by Timothy A. Davis.  Portions are also
copyrighted by William W. Hager (the {\tt Modify} Module),
and the University of Florida (the {\tt Partition} and {\tt Core} Modules).
All Rights Reserved.

See CHOLMOD/Doc/License.txt for the license.
CHOLMOD is also available under other licenses that permit its use in
proprietary applications; contact the authors for details.
See http://www.suitesparse.com for the code and all documentation,
including this User Guide.

\newpage
\tableofcontents

%-------------------------------------------------------------------------------
\newpage \section{Overview}
%-------------------------------------------------------------------------------

CHOLMOD is a set of ANSI C routines for solving systems of linear
equations, $\m{Ax}=\m{b}$, when $\m{A}$ is sparse and symmetric positive definite,
and $\m{x}$ and $\m{b}$ can be either sparse or dense.\footnote{Some support
is provided for symmetric indefinite matrices.}
Complex matrices are supported, in two different formats.
CHOLMOD includes high-performance left-looking supernodal factorization
and solve methods \cite{NgPeyton91b},
based on LAPACK \cite{LAPACK} and the BLAS \cite{ACM679a}.
After a matrix is factorized, its factors can be updated or downdated using
the techniques described by Davis and Hager
in \cite{DavisHager99,DavisHager01,DavisHager05}.
Many additional sparse matrix operations are provided, for both
symmetric and unsymmetric matrices (square or rectangular), including
sparse matrix multiply, add, transpose, permutation, scaling,
norm, concatenation, sub-matrix access, and converting to alternate data structures.
Interfaces to many ordering methods are provided, including minimum degree
(AMD \cite{AmestoyDavisDuff96,AmestoyDavisDuff03},
COLAMD \cite{DavisGilbertLarimoreNg00_algo,DavisGilbertLarimoreNg00}),
constrained minimum degree (CSYMAMD, CCOLAMD, CAMD), and
graph-partitioning-based nested dissection (METIS \cite{KarypisKumar98}).
Most of its operations are available within MATLAB via mexFunction interfaces.

    CHOLMOD also includes a non-supernodal $\m{LDL}^T$ factorization method
    that can factorize symmetric indefinite matrices if all of their
    leading submatrices are well-conditioned ($\m{D}$ is diagonal).

A pair of articles on CHOLMOD has been submitted to the ACM Transactions
on Mathematical Software:
\cite{ChenDavisHagerRajamanickam06,DavisHager06}.

CHOLMOD 1.0 replaces {\tt chol} (the sparse case), {\tt symbfact}, and {\tt etree}
in MATLAB 7.2 (R2006a), and is used for {\tt x=A}$\backslash${\tt b}
when {\tt A} is symmetric positive definite \cite{GilbertMolerSchreiber}.
It will replace {\tt sparse} in a future version of MATLAB.

The C-callable CHOLMOD library consists of 133 user-callable routines and one
include file.  Each routine comes in two versions, one for {\tt int} integers
and another for {\tt long}.  Many of the routines can support either real or
complex matrices, simply by passing a matrix of the appropriate type.

Nick Gould, Yifan Hu, and Jennifer Scott have independently tested CHOLMOD's
performance, comparing it with nearly a dozen or so other solvers
\cite{GouldHuScott05,GouldHuScott05b}.  Its performance was quite competitive.

%-------------------------------------------------------------------------------
\newpage \section{Primary routines and data structures}
%-------------------------------------------------------------------------------

Five primary CHOLMOD routines are required to factorize $\m{A}$ or $\m{AA}\tr$
and solve the related system $\m{Ax}=\m{b}$ or $\m{AA}\tr\m{x}=\m{b}$,
for either the real or complex cases:
\begin{enumerate}
\item {\tt cholmod\_start}:
    This must be the first call to CHOLMOD.

\item {\tt cholmod\_analyze}:
    Finds a fill-reducing ordering, and performs the symbolic factorization,
    either simplicial (non-supernodal) or supernodal.

\item {\tt cholmod\_factorize}:
    Numerical factorization, either simplicial or supernodal, $\m{LL}\tr$ or $\m{LDL}\tr$
    using either the symbolic factorization from {\tt cholmod\_analyze} or the numerical
    factorization from a prior call to {\tt cholmod\_factorize}.

\item {\tt cholmod\_solve}:
    Solves $\m{Ax}=\m{b}$, or many other related systems, where $\m{x}$ and
    $\m{b}$ are dense matrices.  The {\tt cholmod\_spsolve} routine handles
    the sparse case.  Any mixture of real and complex $\m{A}$ and $\m{b}$ are
    allowed.

\item {\tt cholmod\_finish}:
    This must be the last call to CHOLMOD.
\end{enumerate}

Additional routines are also required to create and destroy
the matrices $\m{A}$, $\m{x}$, $\m{b}$, and the $\m{LL}\tr$ or $\m{LDL}\tr$ factorization.
CHOLMOD has five kinds of data structures, referred to as objects and implemented
as pointers to {\tt struct}'s:

\begin{enumerate}
\item {\tt cholmod\_common}:  parameter settings, statistics, and workspace
    used internally by CHOLMOD.
    See Section~\ref{cholmod_common} for details.

\item {\tt cholmod\_sparse}:  a sparse matrix in compressed-column form,
    either pattern-only, real, complex, or ``zomplex.''  In its basic form,
    the matrix {\tt A} contains:
    \begin{itemize}
	\item {\tt A->p}, an integer array of size {\tt A->ncol+1}.
	\item {\tt A->i}, an integer array of size {\tt A->nzmax}.
	\item {\tt A->x}, a {\tt double} array of size {\tt A->nzmax} or twice that for the complex case.
	    This is compatible with the Fortran and ANSI C99 complex data type.
	\item {\tt A->z}, a {\tt double} array of size {\tt A->nzmax} if {\tt A} is zomplex.
	    A zomplex matrix has a {\tt z} array, thus the name.
	    This is compatible with the MATLAB representation of complex matrices.
    \end{itemize}
    For all four types of matrices, the row indices of entries of column {\tt j}
    are located in {\tt A->i [A->p [j] ... A->p [j+1]-1]}.
    For a real matrix, the corresponding numerical values are in {\tt A->x} at the same location.
    For a complex matrix, the entry whose row index is {\tt A->i [p]} is contained in
	{\tt A->x [2*p]} (the real part) and {\tt A->x [2*p+1]} (the imaginary part).
    For a zomplex matrix, the real part is in {\tt A->x [p]} and imaginary part is in {\tt A->z [p]}.
    See Section~\ref{cholmod_sparse} for more details.

\item {\tt cholmod\_factor}:
    A symbolic or numeric factorization, either real, complex, or zomplex.
    It can be either an $\m{LL}\tr$ or $\m{LDL}\tr$ factorization, and either
    simplicial or supernodal.  You will normally not need to examine its contents.
    See Section~\ref{cholmod_factor} for more details.

\item {\tt cholmod\_dense}:
    A dense matrix, either real, complex or zomplex, in column-major order.
    This differs from the row-major convention used in C.  A dense matrix {\tt X} contains
    \begin{itemize}
	\item {\tt X->x}, a double array of size {\tt X->nzmax} or twice that for the complex case.
	\item {\tt X->z}, a double array of size {\tt X->nzmax} if {\tt X} is zomplex.
    \end{itemize}
    For a real dense matrix $x_{ij}$ is {\tt X->x [i+j*d]} where {\tt d = X->d} is the leading dimension of {\tt X}.
    For a complex dense matrix, the real part of $x_{ij}$ is {\tt X->x [2*(i+j*d)]} and the imaginary part is {\tt X->x [2*(i+j*d)+1]}.
    For a zomplex dense matrix, the real part of $x_{ij}$ is {\tt X->x [i+j*d]} and the imaginary part is {\tt X->z [i+j*d]}.
    Real and complex dense matrices can be passed to LAPACK and the BLAS.
    See Section~\ref{cholmod_dense} for more details.

\item {\tt cholmod\_triplet}:
    CHOLMOD's sparse matrix ({\tt cholmod\_sparse}) is the primary input for nearly all CHOLMOD
    routines, but it can be difficult for the user to construct.
    A simpler method of creating a sparse matrix is to first create a {\tt cholmod\_triplet} matrix,
    and then convert it to a {\tt cholmod\_sparse} matrix via
    the {\tt cholmod\_triplet\_to\_sparse} routine.
    In its basic form, the triplet matrix {\tt T} contains
    \begin{itemize}
	\item {\tt T->i} and {\tt T->j}, integer arrays of size {\tt T->nzmax}.
	\item {\tt T->x}, a double array of size {\tt T->nzmax} or twice that for the complex case.
	\item {\tt T->z}, a double array of size {\tt T->nzmax} if {\tt T} is zomplex.
    \end{itemize}
    The {\tt k}th entry in the data structure has row index
    {\tt T->i [k]} and column index {\tt T->j [k]}.
    For a real triplet matrix, its numerical value is {\tt T->x [k]}.
    For a complex triplet matrix, its real part is {\tt T->x [2*k]} and its imaginary part is {\tt T->x [2*k+1]}.
    For a zomplex matrix, the real part is {\tt T->x [k]} and imaginary part is {\tt T->z [k]}.
    The entries can be in any order, and duplicates are permitted.
    See Section~\ref{cholmod_triplet} for more details.

\end{enumerate}

Each of the five objects has a routine in CHOLMOD to create and destroy it.
CHOLMOD provides many other operations on these objects as well.
A few of the most important ones are illustrated in the sample program in the
next section.

%-------------------------------------------------------------------------------
\newpage \section{Simple example program}
%-------------------------------------------------------------------------------

\input{_simple.tex}
The {\tt Demo/cholmod\_simple.c} program illustrates the
basic usage of CHOLMOD.  It reads a triplet matrix from a file
(in Matrix Market format), converts it into a sparse matrix,
creates a linear system, solves it, and prints the norm of the residual.

See the {\tt CHOLMOD/Demo/cholmod\_demo.c} program for a more elaborate
example, and \newline
{\tt CHOLMOD/Demo/cholmod\_l\_demo.c} for its {\tt long} integer version.

%-------------------------------------------------------------------------------
\newpage \section{Installation of the C-callable library}
\label{Install}
%-------------------------------------------------------------------------------

CHOLMOD requires a suite of external packages, many of which are distributed
along with CHOLMOD, but three of which are not.  Those included with CHOLMOD are:
\begin{itemize}
\item AMD: an approximate minimum degree ordering algorithm,
    by Tim Davis, Patrick Amestoy, and Iain Duff
    \cite{AmestoyDavisDuff96,AmestoyDavisDuff03}.
\item COLAMD: an approximate column minimum degree ordering algorithm,
    by Tim Davis, Stefan Larimore, John Gilbert, and Esmond Ng
    \cite{DavisGilbertLarimoreNg00_algo,DavisGilbertLarimoreNg00}.
\item CCOLAMD: a constrained approximate column minimum degree ordering
    algorithm,
    by Tim Davis and Siva Rajamanickam, based directly on COLAMD.
    This package is not required if CHOLMOD is compiled with the
    {\tt -DNCAMD} flag.
\item CAMD: a constrained approximate minimum degree ordering
    algorithm,
    by Tim Davis and Yanqing Chen, based directly on AMD.
    This package is not required if CHOLMOD is compiled with the
    {\tt -DNCAMD} flag.
\item {\tt SuiteSparse\_config}:
    a single place where all sparse matrix packages authored
    or co-authored by Davis are configured.  Also includes a version of the
    {\tt xerbla} routine for the BLAS.
\end{itemize}

Three other packages are required for optimal performance:
\begin{itemize}
\item {\tt METIS 5.1.0}: a graph partitioning package by George Karypis,
    Univ. of Minnesota.  Not needed if {\tt -DNPARTITION} is used.
    See http://www-users.cs.umn.edu/$\sim$karypis/metis.
\item BLAS: the Basic Linear Algebra Subprograms.
    Not needed if {\tt -DNSUPERNODAL} is used.
    See http://www.netlib.org for the reference BLAS (not meant for production
    use).  For Kazushige Goto's optimized BLAS (highly recommended for CHOLMOD)
    see \newline
    http://www.tacc.utexas.edu/$\sim$kgoto/ or
    http://www.cs.utexas.edu/users/flame/goto/.
    I recommend that you avoid the Intel MKL BLAS; one recent
    version returns NaN's, where both the Goto BLAS and the standard
    Fortran reference BLAS return the correct answer.
    See {\tt CHOLMOD/README} for more information.
\item LAPACK: the Basic Linear Algebra Subprograms.
    Not needed if {\tt -DNSUPERNODAL} is used.
    See http://www.netlib.org.
\item CUDA BLAS:  CHOLMOD can exploit an NVIDIA GPU by using the CUDA BLAS
    for large supernodes.  This feature is new to CHOLMOD v2.0.0.
\end{itemize}

You must first obtain and install LAPACK, and the BLAS.
METIS 5.1.0 is optional; a copy of it is in {\tt SuiteSparse}.
System-dependent configurations in the
{\tt SuiteSparse\_config/SuiteSparse\_config.mk} file.
This file has been changed in CHOLMOD version 3.0.7;
it now automatically detects your system, your BLAS, and whether or not
you have CUDA installed, and whether or not you have METIS 5.1.0.
You still may need to edit it to refine the
compilation parameters for your particular system, but it is likely it
will work unmodified.

\noindent
Here are some of the various parameters that you can control in your
{\tt SuiteSparse\_config/SuiteSparse\_config.mk} file.
You can set them without editing that file, simply by including
them on your {\tt make} command.  For example, to {\tt -lmyblas},
use {\tt make BLAS=-lmyblas}.  For a complete list, including
their default values, do {\tt make config}.
\begin{itemize}
\item {\tt CC = } your C compiler, such as {\tt cc}.
\item {\tt CF = } optimization flags, such as {\tt -O}.
\item {\tt RANLIB = } your system's {\tt ranlib} program, if needed.
\item {\tt ARCHIVE =} the command to create a library (such as {\tt ar}).
\item {\tt RM =} the command to delete a file.
\item {\tt MV =} the command to rename a file.
\item {\tt F77 =} the command to compile a Fortran program (optional).
\item {\tt F77FLAGS =} the Fortran compiler flags (optional).
\item {\tt F77LIB =} the Fortran libraries (optional).
\item {\tt LDLIBS = } basic libraries, such as {\tt -lm}.
\item {\tt BLAS =} your BLAS library.
\item {\tt LAPACK =} your LAPACK library.
\item {\tt METIS\_PATH =} the path to your copy of the METIS 5.1.0 source code.
\item {\tt METIS =} your METIS library.
\item {\tt GPU\_CONFIG = } configuration settings specific to the CUDA BLAS.
\item {\tt CHOLMOD\_CONFIG = } configuration settings specific to CHOLMOD.
\end{itemize}

\noindent
CHOLMOD's specific settings are given by the {\tt CHOLMOD\_CONFIG} string:
\begin{itemize}
\item {\tt -DNCHECK}:	    do not include the Check module.
\item {\tt -DNCHOLESKY}:    do not include the Cholesky module.
\item {\tt -DNPARTITION}:   do not include the interface to METIS in
                            the Partition module.
\item {\tt -DCAMD}:         do not include the interfaces to CAMD, CCOLAMD,
                            and CSYMAMD in the Partition module.
\item {\tt -DNMATRIXOPS}:   do not include the MatrixOps module.
                            Note that the Demo requires the MatrixOps module.
\item {\tt -DNMODIFY}:	    do not include the Modify module.
\item {\tt -DNSUPERNODAL}:  do not include the Supernodal module.
\item {\tt -DNPRINT}:	    do not print anything.
\item {\tt -D'LONGBLAS=long'} or {\tt -DLONGBLAS='long long'}
			    defines the integers used by LAPACK
                            and the BLAS (defaults to {\tt int}).
\item {\tt -DNSUNPERF}:	    for Solaris only.  If defined, do not use the
                            Sun Performance Library.
\item {\tt -DNLARGEFILE}:   CHOLMOD now assumes support for large files (2GB or
                            larger).  If this causes problems, you can compile
                            CHOLMOD with -DNLARGEFILE.  To use large files, you
                            should {\tt \#include "cholmod.h"} (or at least
                            {\tt \#include "cholmod\_io64.h"}) before any other
                            {\tt \#include} statements, in your application
                            that uses CHOLMOD.  You may need to use {\tt
                            fopen64} to create a file pointer to pass to
                            CHOLMOD, if you are using a non-gcc compiler.
\end{itemize}

Type {\tt make} in the {\tt CHOLMOD} directory.  The AMD,
COLAMD, CAMD, CCOLAMD, and {\tt CHOLMOD} libraries will be compiled,
as will the C version of the null-output {\tt xerbla} routine in case you need it.
No Fortran compiler is required in this case.  A short demo program will
be compiled and tested on a few matrices.  The residuals should all be small.
Compare your output with the {\tt CHOLMOD/Demo/make.out} file.

CHOLMOD is now ready for use in your own applications.  You must link
your programs with the
{\tt libcholmod.*},
{\tt libamd.*},
{\tt libcolamd.*},
LAPACK,
and
BLAS libraries,
as well as the {\tt xerbla} library if you need it
({\tt SuiteSparse\_config/xerbla/libcerbla.*} for the C version or
\newline
 {\tt SuiteSparse\_config/xerbla/libxerbla.*} for the Fortran version).
Unless you use {\tt -DNPARTITION}, you must also link with the METIS 5.1.0
library.
Unless {\tt -DNCAMD} is present at compile time,
you must link with {\tt CAMD/libcamd.*}, and {\tt CCOLAMD/libccolamd.*}.

The {\tt make} command now copies all of these libraries and include files
into a single place: {\tt SuiteSparse/lib} and {\tt SuiteSparse/include}.
To tell your your compiler where to find them, use
{\tt -LSuiteSparse/lib} and {\tt -ISuiteSparse/include}.

To install CHOLMOD in /usr/local/lib and /usr/local/include, do
{\tt make install}.
If you do this, youdo not need the {\tt -L} and {\tt -I} option when compiling
your program.
Documentation is also installed in /usr/local/doc.
The installation location can be changed at the {\tt make} command line; see the
{\tt SuiteSparse/README.txt} file for details.
To remove CHOLMOD, do {\tt make uninstall}.

%-------------------------------------------------------------------------------
\newpage \section{Using CHOLMOD in MATLAB}
%-------------------------------------------------------------------------------

CHOLMOD includes a set of m-files and mexFunctions in the CHOLMOD/MATLAB
directory.  The following functions are provided:

\vspace{0.1in}
\begin{tabular}{ll}
\hline
{\tt analyze}	    & order and analyze a matrix \\
{\tt bisect}	    & find a node separator \\
{\tt chol2}	    & same as {\tt chol} \\
{\tt cholmod2}	    & same as {\tt x=A}$\backslash${\tt b} if {\tt A} is symmetric positive definite \\
{\tt cholmod\_demo} & a short demo program \\
{\tt cholmod\_make} & compiles CHOLMOD for use in MATLAB \\
{\tt etree2}	    & same as {\tt etree} \\
{\tt graph\_demo}   & graph partitioning demo \\
{\tt lchol}	    & {\tt L*L'} factorization \\
{\tt ldlchol}	    & {\tt L*D*L'} factorization \\
{\tt ldl\_normest}  & estimate {\tt norm(A-L*D*L')} \\
{\tt ldlsolve}	    & {\tt x = L'}$\backslash${\tt (D}$\backslash${\tt (L}$\backslash${\tt b))} \\
{\tt ldlsplit}	    & split the output of {\tt ldlchol} into {\tt L} and {\tt D} \\
{\tt ldlupdate}	    & update/downdate an {\tt L*D*L'} factorization \\
{\tt ldlrowmod}	    & add/delete a row from an {\tt L*D*L'} factorization \\
{\tt metis}	    & interface to {\tt METIS\_NodeND} ordering \\
{\tt mread}	    & read a sparse or dense Matrix Market file \\
{\tt mwrite}	    & write a sparse or dense Matrix Market file \\
{\tt nesdis}	    & CHOLMOD's nested dissection ordering \\
{\tt resymbol}	    & recomputes the symbolic factorization \\
{\tt sdmult}	    & {\tt S*F} where {\tt S} is sparse and {\tt F} is dense \\
{\tt spsym}	    & determine symmetry \\
{\tt sparse2}	    & same as {\tt sparse} \\
{\tt symbfact2}	    & same as {\tt symbfact} \\
\hline
\end{tabular}

\vspace{0.1in}\noindent
Each function is described in the next sections.

\newpage
\subsection{{\tt analyze}: order and analyze}					\input{_analyze_m.tex}
\subsection{{\tt bisect}: find a node separator}				\input{_bisect_m.tex}
\subsection{{\tt chol2}: same as {\tt chol}}					\input{_chol2_m.tex}
\newpage
\subsection{{\tt cholmod2}: supernodal backslash}				\input{_cholmod2_m.tex}
\newpage
\subsection{{\tt cholmod\_demo}: a short demo program}				\input{_cholmod_demo_m.tex}
\subsection{{\tt cholmod\_make}: compile CHOLMOD in MATLAB}			\input{_cholmod_make_m.tex}
\newpage
\subsection{{\tt etree2}: same as {\tt etree}}					\input{_etree2_m.tex}
\newpage
\subsection{{\tt graph\_demo}: graph partitioning demo}				\input{_graph_demo_m.tex}
\newpage
\subsection{{\tt lchol}: $\m{LL}\tr$ factorization}				\input{_lchol_m.tex}
\subsection{{\tt ldlchol}: $\m{LDL}\tr$ factorization}				\input{_ldlchol_m.tex}
\newpage
\subsection{{\tt ldlsolve}: solve using an $\m{LDL}\tr$ factorization}		\input{_ldlsolve_m.tex}
\subsection{{\tt ldlsplit}: split an $\m{LDL}\tr$ factorization}		\input{_ldlsplit_m.tex}
\newpage
\subsection{{\tt ldlupdate}: update/downdate an $\m{LDL}\tr$ factorization}	\input{_ldlupdate_m.tex}
\newpage
\subsection{{\tt ldlrowmod}: add/delete a row from an $\m{LDL}\tr$ factorization}	\input{_ldlrowmod_m.tex}
\newpage
\subsection{{\tt mread}: read a sparse or dense matrix from a Matrix Market file}\input{_mread_m.tex}
\subsection{{\tt mwrite}: write a sparse or dense matrix to a Matrix Market file}	\input{_mwrite_m.tex}
\newpage
\subsection{{\tt metis}: order with METIS}					\input{_metis_m.tex}
\newpage
\subsection{{\tt nesdis}: order with CHOLMOD nested dissection}			\input{_nesdis_m.tex}
\newpage
\subsection{{\tt resymbol}: re-do symbolic factorization}			\input{_resymbol_m.tex}
\subsection{{\tt sdmult}: sparse matrix times dense matrix}			\input{_sdmult_m.tex}
\newpage
\subsection{{\tt spsym}: determine symmetry}				\input{_spsym_m.tex}
\newpage
\subsection{{\tt sparse2}: same as {\tt sparse}}				\input{_sparse2_m.tex}
\newpage
\subsection{{\tt symbfact2}: same as {\tt symbfact}}				\input{_symbfact2_m.tex}

%-------------------------------------------------------------------------------
\newpage \section{Installation for use in MATLAB}
%-------------------------------------------------------------------------------

%-------------------------------------------------------------------------------
\subsection{{\tt cholmod\_make}: compiling CHOLMOD in MATLAB}
%-------------------------------------------------------------------------------

This is the preferred method, since it allows METIS to be reconfigured to
use the MATLAB memory-management functions instead of {\tt malloc} and {\tt free};
this avoids the issue of METIS terminating MATLAB if it runs out of memory.
It is also simpler for Windows users, who do not have the {\tt make}
command (unless you obtain a copy of {\tt Cygwin}).

Start MATLAB, {\tt cd} to the {\tt CHOLMOD/MATLAB} directory, and
type {\tt cholmod\_make} in the MATLAB command window.  This will compile
the MATLAB interfaces for AMD, COLAMD, CAMD, CCOLAMD, METIS, and CHOLMOD.

%-------------------------------------------------------------------------------
\newpage \section{Using CHOLMOD with GPU acceleration}
%-------------------------------------------------------------------------------

Starting with CHOLMOD v2.0.0, it is possible to accelerate the numerical
factorization phase of CHOLMOD using NVIDIA GPUs.  Due to the large
computational capability of the GPUs, enabling this capability can result in
significant performance improvements.  Similar to CPU processing, the GPU is
better able to accelerate the dense math associated with larger supernodes.
Hence the GPU will provide more significant performance improvements for larger
matrices that have more, larger supernodes.

In CHOLMOD v2.3.0 this GPU capability has been improved to provide a
significant increase in performance and the interface has been expanded to make
the use of GPUs more flexible.  CHOLMOD can take advantage of a single NVIDIA
GPU that supports CUDA and has at least 64MB of memory.  (But substantially
more memory, typically about 3 GB, is recommended for best performance.)

Only the {\tt long} integer version of CHOLMOD can leverage GPU acceleration.

%-------------------------------------------------------------------------------
\subsection{Compiling CHOLMOD with GPU support}
%-------------------------------------------------------------------------------

In order to support GPU processing, CHOLMOD must be compiled with the
preprocessor macro {\tt GPU\_BLAS} defined.  All GPU code is conditional upon
this macro.  As well, the environment variable {\tt CUDA\_ROOT} must be defined
and point to the installation of the CUDA toolkit to be used for compilation of
CHOLMOD.  Typically this would be {\tt /usr/local/cuda}.

The {\tt SuiteSparse\_config.mk} should automatically detect if you have
CUDA installed on your system.  If so, then it will do everything for you
without the need to edit the {\tt SuiteSparse\_config.mk} file.

%-------------------------------------------------------------------------------
\subsection{Enabling GPU acceleration in CHOLMOD}
%-------------------------------------------------------------------------------

Even if compiled with GPU support, in CHOLMOD v.2.3.0, GPU processing is not
enabled by default and must be specifically requested.  There are two ways to
do this, either in the code calling CHOLMOD or using environment variables.

The code author can specify the use of GPU processing with the {\tt
Common->useGPU} variable.  If this is set to {\tt 1}, CHOLMOD will attempt to
use the GPU.  If this is set to {\tt 0} the use of the GPU will be prohibited.
If this is set to {\tt -1}, which is the default case, then the environment
variables (following paragraph) will be queried to determine if the GPU is to
be used.  Note that the default value of {\tt -1} is set when {\tt
cholmod\_start(Common)} is called, so the code author must set {\tt
Common->useGPU} after calling {\tt cholmod\_start}.

Alternatively, or if it is not possible to modify the code calling CHOLMOD, GPU
processing can invoked using the {\tt CHOLMOD\_USE\_GPU} environment variable.
This makes it possible for any CHOLMOD user to invoke GPU processing even if
the author of the calling program did not consider this.  The interpretation of
the environment variable {\tt CHOLMOD\_USE\_GPU} is that if the string
evaluates to an integer other than zero, GPU processing will be enabled.  Note
that the setting of {\tt Common->useGPU} takes precedence and the environment
variable {\tt CHOLMOD\_USE\_GPU} will only be queried if {\tt Common->useGPU =
-1}.

Note that in either case, if GPU processing is requested, but there is no GPU
present, CHOLMOD will continue using the CPU only.  Consequently it is always
safe to request GPU processing.

%-------------------------------------------------------------------------------
\newpage \subsection{Adjustable parameters}
%-------------------------------------------------------------------------------

There are a number of parameters that have been added to CHOLMOD to control GPU
processing.  All of these have appropriate defaults such that GPU processing
can be used without any modification.  However, for any particular combination
of CPU/GPU, better performance might be obtained by adjusting these parameters.

\bigskip

From {\tt t\_cholmod\_gpu.c}

\begin{quote}
  {\tt CHOLMOD\_ND\_ROW\_LIMIT} : Minimum number of rows required in a
  descendant supernode to be eligible for GPU processing during supernode
  assembly


  {\tt CHOLMOD\_ND\_COL\_LIMIT} : Minimum number of columns in a descendant
  supernode to be eligible for GPU processing during supernode assembly


  {\tt CHOLMOD\_POTRF\_LIMIT} : Minimum number of columns in a supernode to be
  eligible for {\tt POTRF} and {\tt TRSM} processing on the GPU


  {\tt CHOLMOD\_GPU\_SKIP} : Number of small descendant supernodes to assembled
  on the CPU before querying if the GPU is needs more descendant supernodes
  queued 

\end{quote}

From {\tt cholmod\_core.h}

\begin{quote}
  {\tt CHOLMOD\_HOST\_SUPERNODE\_BUFFERS} : Number of buffers in which to queue
  descendant supernodes for GPU processing

\end{quote}

Programatically

\begin{quote}

  {\tt Common->maxGpuMemBytes} : Specifies the maximum amount of memory, in
  bytes, that CHOLMOD can allocate on the GPU.  If this parameter is not set,
  CHOLMOD will allocate as much GPU memory as possible.  Hence, the purpose of
  this parameter is to restrict CHOLMOD's GPU memory use so that CHOLMOD can be
  used simultaneously with other codes that also use GPU acceleration and
  require some amount of GPU memory.  If the specified amount of GPU memory is
  not allocatable, CHOLMOD will allocate the available memory and continue.

  {\tt Common->maxGpuMemFraction} : Entirely similar to {\tt
  Common->maxGpuMemBytes} but with the memory specified as a fraction of total
  GPU memory.  Note that if both {\tt maxGpuMemBytes} and {\tt
  maxGpuMemFraction} are specified, whichever results in the minimum amount of
  memory will be used.

\end{quote}

Environment variables

\begin{quote}
  {\tt CHOLMOD\_GPU\_MEM\_BYTES} : Environment variable with a meaning
  equivalent to {\tt Common->maxGpuMemBytes}.  This will only be queried if
  {\tt Common->useGPU = -1}.

  {\tt CHOLMOD\_GPU\_MEM\_FRACTION} : Environment variable with a meaning
  equivalent to {\tt Common->maxGpuMemFraction}.  This will only be queried if
  {\tt Common->useGPU = -1}.

\end{quote}

%-------------------------------------------------------------------------------
\newpage \section{Integer and floating-point types, and notation used}
%-------------------------------------------------------------------------------

CHOLMOD supports both {\tt int} and {\tt long} integers.  CHOLMOD
routines with the prefix {\tt cholmod\_} use {\tt int} integers,
{\tt cholmod\_l\_} routines use {\tt long}.  All floating-point
values are {\tt double}.

The {\tt long} integer is redefinable, via {\tt SuiteSparse\_config.h}.
That file defines a C preprocessor token {\tt SuiteSparse\_long} which is
{\tt long} on all systems except for Windows-64, in which case it is
defined as {\tt \_\_int64}.  The intent is that with suitable compile-time
switches, {\tt int} is a 32-bit integer and {\tt SuiteSparse\_long} is a 64-bit
integer.  The term {\tt long} is used to describe the latter
integer throughout this document (except in the prototypes).

Two kinds of complex matrices are supported: complex and zomplex.
A complex matrix is held in a manner that is compatible with the
Fortran and ANSI C99 complex data type.  A complex array of size {\tt n}
is a {\tt double} array {\tt x} of size {\tt 2*n}, with the real and imaginary
parts interleaved (the real part comes first, as a {\tt double}, followed the
imaginary part, also as a {\tt double}.  Thus, the real part of the {\tt k}th
entry is {\tt x[2*k]} and the imaginary part is {\tt x[2*k+1]}.

A zomplex matrix of size {\tt n} stores its real part in one
{\tt double} array of size {\tt n} called {\tt x} and its imaginary part
in another {\tt double} array of size {\tt n} called {\tt z} (thus the
name ``zomplex'').  This also how MATLAB stores its complex matrices.
The real part of the {\tt k}th entry is {\tt x[k]} and the imaginary part is
{\tt z[k]}.

Unlike {\tt UMFPACK}, the same routine name in CHOLMOD is used for pattern-only,
real, complex, and zomplex matrices.  For example, the statement
\begin{verbatim}
    C = cholmod_copy_sparse (A, &Common) ;
\end{verbatim}
creates a copy of a pattern, real, complex, or zomplex sparse matrix {\tt A}.
The xtype (pattern, real, complex, or zomplex) of the resulting sparse matrix {\tt C}
is the same as {\tt A} (a pattern-only sparse matrix contains no floating-point
values).  In the above case, {\tt C} and {\tt A} use {\tt int} integers.
For {\tt long} integers, the statement would become:
\begin{verbatim}
    C = cholmod_l_copy_sparse (A, &Common) ;
\end{verbatim}
The last parameter of all CHOLMOD routines is always {\tt \&Common},
a pointer to the
{\tt cholmod\_common} object, which contains parameters, statistics,
and workspace used throughout CHOLMOD.

The {\tt xtype} of a CHOLMOD object (sparse matrix, triplet matrix, dense
matrix, or factorization) determines whether it is pattern-only,
real, complex, or zomplex.

The names of the {\tt int} versions are primarily used in this document.
To obtain the name of the {\tt long} version of the same routine, simply
replace {\tt cholmod\_} with {\tt cholmod\_l\_}.

MATLAB matrix notation is used throughout this document and in
the comments in the CHOLMOD code itself.  If you are not familiar with
MATLAB, here is a short introduction to the notation, and a few
minor variations used in CHOLMOD:

\begin{itemize}
    \item {\tt C=A+B} and {\tt C=A*B}, respectively are a matrix add and multiply if both
	{\tt A} and {\tt B} are matrices of appropriate size.  If {\tt A} is
	a scalar, then it is added to or multiplied with every entry in {\tt B}.
    \item {\tt a:b} where {\tt a} and {\tt b} are integers refers to the
	sequence {\tt a}, {\tt a+1}, ... {\tt b}.
    \item {\tt [A B]} and {\tt [A,B]} are the horizontal concatenation of {\tt A} and {\tt B}.
    \item {\tt [A;B]} is the vertical concatenation of {\tt A} and {\tt B}.
    \item {\tt A(i,j)} can refer either to a scalar or a submatrix.
	For example: \newline
	\vspace{0.05in}
	\begin{tabular}{ll}
	\hline
	{\tt A(1,1)} & a scalar. \\
	{\tt A(:,j)} & column {\tt j} of {\tt A}. \\
	{\tt A(i,:)} & row {\tt i} of {\tt A}. \\
	{\tt A([1 2], [1 2])} & a 2-by-2 matrix containing the 2-by-2 leading minor of {\tt A}. \\
	\hline
	\end{tabular} \newline
	\vspace{0.1in}
	If {\tt p} is a permutation of {\tt 1:n}, and {\tt A} is {\tt n}-by-{\tt n},
	then {\tt A(p,p)} corresponds to the permuted matrix $\m{PAP}\tr$.
    \item {\tt tril(A)} is the lower triangular part of {\tt A}, including the diagonal.
    \item {\tt tril(A,k)} is the lower triangular part of {\tt A}, including entries
	on and below the $k$th diagonal.
    \item {\tt triu(A)} is the upper triangular part of {\tt A}, including the diagonal.
    \item {\tt triu(A,k)} is the upper triangular part of {\tt A}, including entries
	on and above the $k$th diagonal.
    \item {\tt size(A)} returns the dimensions of {\tt A}.
    \item {\tt find(x)} if {\tt x} is a vector returns a list of indices {\tt i}
	for which {\tt x(i)} is nonzero.
    \item {\tt A'} is the transpose of {\tt A} if {\tt A} is real, or
	the complex conjugate transpose if {\tt A} is complex.
    \item {\tt A.'} is the array transpose of {\tt A}.
    \item {\tt diag(A)} is the diagonal of {\tt A} if {\tt A} is a matrix.
    \item {\tt C=diag(s)} is a diagonal matrix if {\tt s} is a vector,
	with the values of {\tt s} on the diagonal of {\tt C}.
    \item {\tt S=spones(A)} returns a binary matrix {\tt S} with the
	same nonzero pattern of {\tt A}.
    \item {\tt nnz(A)} is the number of nonzero entries in {\tt A}.
\end{itemize}

\noindent Variations to MATLAB notation used in this document:
\begin{itemize}
    \item CHOLMOD uses 0-based notation (the first entry in the matrix is
	{\tt A(0,0)}).  MATLAB is 1-based.  The context is usually clear.
    \item {\tt I} is the identity matrix.
    \item {\tt A(:,f)}, where {\tt f} is a set of columns, is interpreted
	differently in CHOLMOD, but just for the set named {\tt f}.
	See {\tt cholmod\_transpose\_unsym} for details.
\end{itemize}

%-------------------------------------------------------------------------------
\newpage \section{The CHOLMOD Modules, objects, and functions}
\label{Modules}
%-------------------------------------------------------------------------------

CHOLMOD contains a total of 133 {\tt int}-based routines (and the same number
of {\tt long} routines), divided into a set of inter-related
Modules.  Each Module contains a set of related functions.  The functions
are divided into two types: Primary and Secondary, to reflect how a user will
typically use CHOLMOD.  Most users will find the Primary routines to be
sufficient to use CHOLMOD in their programs.  Each Module exists as a
sub-directory (a folder for Windows users) within the CHOLMOD directory
(or folder).

\vspace{0.1in}
\noindent There are seven Modules that provide user-callable routines for CHOLMOD.
    \begin{enumerate}
    \item {\tt Core}: basic data structures and definitions
    \item {\tt Check}: prints/checks each of CHOLMOD's objects
    \item {\tt Cholesky}: sparse Cholesky factorization
    \item {\tt Modify}: sparse Cholesky update/downdate and row-add/row-delete
    \item {\tt MatrixOps}: sparse matrix operators (add, multiply, norm, scale)
    \item {\tt Supernodal}: supernodal sparse Cholesky factorization
    \item {\tt Partition}: graph-partitioning-based orderings
    \end{enumerate}

\noindent Two additional Modules are required to compile the CHOLMOD library:
    \begin{enumerate}
    \item {\tt Include}: include files for CHOLMOD and programs that use CHOLMOD
    \item {\tt Lib}: where the CHOLMOD library is built
    \end{enumerate}

\noindent Five additional Modules provide support functions and documentation:
    \begin{enumerate}
    \item {\tt Demo}: simple programs that illustrate the use of CHOLMOD
    \item {\tt Doc}: documentation (including this document)
    \item {\tt MATLAB}: CHOLMOD's interface to MATLAB
    \item {\tt Tcov}: an exhaustive test coverage (requires Linux or Solaris)
    \item {\tt Valgrind}: runs the {\tt Tcov} test under {\tt valgrind} (requires Linux)
    \end{enumerate}

%-------------------------------------------------------------------------------
\newpage \subsection{{\tt Core} Module: basic data structures and definitions}
%-------------------------------------------------------------------------------

CHOLMOD includes five basic objects, defined in the {\tt Core} Module.
The {\tt Core Module} provides basic operations for these objects
and is required by all six other CHOLMOD library Modules:

\subsubsection{{\tt cholmod\_common}: parameters, statistics, and workspace}
    You must call {\tt cholmod\_start} before calling any other
    CHOLMOD routine, and you must call {\tt cholmod\_finish} as your
    last call to CHOLMOD (with the exception of
    {\tt cholmod\_print\_common} and {\tt cholmod\_check\_common}
    in the {\tt Check} Module).
    Once the {\tt cholmod\_common} object is initialized,
    the user may modify CHOLMOD's parameters held in this object,
    and obtain statistics on CHOLMOD's activity.

\vspace{0.1in}
\noindent Primary routines for the {\tt cholmod\_common} object:
% 2
    \begin{itemize}
    \item {\tt cholmod\_start}: the first call to CHOLMOD.
    \item {\tt cholmod\_finish}: the last call to CHOLMOD (frees workspace in the {\tt cholmod\_common} object).
    \end{itemize}

\noindent Secondary routines for the {\tt cholmod\_common} object:
% 9
    \begin{itemize}
    \item {\tt cholmod\_defaults}: restores default parameters
    \item {\tt cholmod\_maxrank}: determine maximum rank for update/downdate.
    \item {\tt cholmod\_allocate\_work}: allocate workspace.
    \item {\tt cholmod\_free\_work}: free workspace.
    \item {\tt cholmod\_clear\_flag}: clear {\tt Flag} array.
    \item {\tt cholmod\_error}: called when CHOLMOD encounters and error.
    \item {\tt cholmod\_dbound}: bounds the diagonal of $\m{L}$ or $\m{D}$.
    \item {\tt cholmod\_hypot}: compute {\tt sqrt(x*x+y*y)} accurately.
    \item {\tt cholmod\_divcomplex}: complex divide.
    \end{itemize}

%-------------------------------------------------------------------------------
\newpage \subsubsection{{\tt cholmod\_sparse}: a sparse matrix in compressed column form}
%-------------------------------------------------------------------------------
    A sparse matrix {\tt A} is held in compressed column form.  In the basic
    type (``packed,'' which corresponds to how MATLAB stores its sparse
    matrices), and {\tt nrow}-by-{\tt ncol} matrix with {\tt nzmax} entries
    is held in three arrays: {\tt p} of size {\tt ncol+1},
    {\tt i} of size {\tt nzmax}, and {\tt x} of size {\tt nzmax}.
    Row indices of nonzero entries in column {\tt j} are held in
    {\tt i [p[j] ... p[j+1]-1]}, and their corresponding numerical values
    are held in {\tt x [p[j] ... p[j+1]-1]}.  The first column starts at
    location zero ({\tt p[0]=0}).
    There may be no duplicate entries.  Row indices in each column may
    be sorted or unsorted (the {\tt A->sorted} flag must be false if
    the columns are unsorted).  The {\tt A->stype} determines the 
    storage mode: 0 if the matrix is unsymmetric, 1 if the matrix is
    symmetric with just the upper triangular part stored, and -1 if
    the matrix is symmetric with just the lower triangular part stored.

    In ``unpacked'' form, an additional array {\tt nz} of size {\tt ncol}
    is used.  The end of column {\tt j} in {\tt i} and {\tt x}
    is given by {\tt p[j]+nz[j]}.  Columns not need be in any particular
    order ({\tt p[0]} need not be zero), and there may be gaps between
    the columns.

\vspace{0.1in}
\noindent Primary routines for the {\tt cholmod\_sparse} object:
% 2
    \begin{itemize}
    \item {\tt cholmod\_allocate\_sparse}: allocate a sparse matrix
    \item {\tt cholmod\_free\_sparse}: free a sparse matrix
    \end{itemize}

\noindent Secondary routines for the {\tt cholmod\_sparse} object:
% 16
    \begin{itemize}
    \item {\tt cholmod\_reallocate\_sparse}: change the size (number of entries) of a sparse matrix.
    \item {\tt cholmod\_nnz}: number of nonzeros in a sparse matrix.
    \item {\tt cholmod\_speye}: sparse identity matrix.
    \item {\tt cholmod\_spzeros}: sparse zero matrix.
    \item {\tt cholmod\_transpose}: transpose a sparse matrix.
    \item {\tt cholmod\_ptranspose}: transpose/permute a sparse matrix.
    \item {\tt cholmod\_transpose\_unsym}: transpose/permute an unsymmetric sparse matrix.
    \item {\tt cholmod\_transpose\_sym}: transpose/permute a symmetric sparse matrix.
    \item {\tt cholmod\_sort}: sort row indices in each column of a sparse matrix.
    \item {\tt cholmod\_band}: extract a band of a sparse matrix.
    \item {\tt cholmod\_band\_inplace}: remove entries not with a band.
    \item {\tt cholmod\_aat}: {\tt C = A*A'}.
    \item {\tt cholmod\_copy\_sparse}: {\tt C = A}, create an exact copy of a sparse matrix.
    \item {\tt cholmod\_copy}: {\tt C = A}, with possible change of {\tt stype}.
    \item {\tt cholmod\_add}: {\tt C = alpha*A + beta*B}.
    \item {\tt cholmod\_sparse\_xtype}: change the {\tt xtype} of a sparse matrix.
    \end{itemize}

%-------------------------------------------------------------------------------
\newpage \subsubsection{{\tt cholmod\_factor}: a symbolic or numeric factorization}
%-------------------------------------------------------------------------------

    A factor can be in $\m{LL}\tr$ or $\m{LDL}\tr$ form, and either supernodal
    or simplicial form.  In simplicial form, this is very much like a
    packed or unpacked {\tt cholmod\_sparse} matrix.  In supernodal
    form, adjacent columns with similar nonzero pattern are stored as
    a single block (a supernode).

\vspace{0.1in}
\noindent Primary routine for the {\tt cholmod\_factor} object:
% 1
    \begin{itemize}
    \item {\tt cholmod\_free\_factor}: free a factor
    \end{itemize}

\noindent Secondary routines for the {\tt cholmod\_factor} object:
% 8
    \begin{itemize}
    \item {\tt cholmod\_allocate\_factor}: allocate a factor.  You will normally use {\tt cholmod\_analyze} to create a factor.
    \item {\tt cholmod\_reallocate\_factor}: change the number of entries in a factor.
    \item {\tt cholmod\_change\_factor}: change the type of a factor ($\m{LDL}\tr$ to $\m{LL}\tr$, supernodal to simplicial, etc.).
    \item {\tt cholmod\_pack\_factor}: pack the columns of a factor.
    \item {\tt cholmod\_reallocate\_column}: resize a single column of a factor.
    \item {\tt cholmod\_factor\_to\_sparse}: create a sparse matrix copy of a factor.
    \item {\tt cholmod\_copy\_factor}: create a copy of a factor.
    \item {\tt cholmod\_factor\_xtype}: change the xtype of a factor.
    \end{itemize}

%-------------------------------------------------------------------------------
\subsubsection{{\tt cholmod\_dense}: a dense matrix}
%-------------------------------------------------------------------------------
    This consists of a dense array of numerical values and its dimensions.

\vspace{0.1in}
\noindent Primary routines for the {\tt cholmod\_dense} object:
% 2
    \begin{itemize}
    \item {\tt cholmod\_allocate\_dense}: allocate a dense matrix.
    \item {\tt cholmod\_free\_dense}: free a dense matrix.
    \end{itemize}

\vspace{0.1in}
\noindent Secondary routines for the {\tt cholmod\_dense} object:
% 8
    \begin{itemize}
    \item {\tt cholmod\_zeros}: allocate a dense matrix of all zeros.
    \item {\tt cholmod\_ones}: allocate a dense matrix of all ones.
    \item {\tt cholmod\_eye}: allocate a dense identity matrix .
    \item {\tt cholmod\_sparse\_to\_dense}: create a dense matrix copy of a sparse matrix.
    \item {\tt cholmod\_dense\_to\_sparse}: create a sparse matrix copy of a dense matrix.
    \item {\tt cholmod\_copy\_dense}: create a copy of a dense matrix.
    \item {\tt cholmod\_copy\_dense2}: copy a dense matrix (pre-allocated).
    \item {\tt cholmod\_dense\_xtype}: change the {\tt xtype} of a dense matrix.
    \end{itemize}

%-------------------------------------------------------------------------------
\newpage \subsubsection{{\tt cholmod\_triplet}: a sparse matrix in ``triplet'' form}
%-------------------------------------------------------------------------------
    The {\tt cholmod\_sparse} matrix is the basic sparse matrix used in
    CHOLMOD, but it can be difficult for the user to construct.  It also
    does not easily support the inclusion of new entries in the matrix.
    The {\tt cholmod\_triplet} matrix is provided to address these issues.
    A sparse matrix in triplet form consists of three arrays of size
    {\tt nzmax}: {\tt i}, {\tt j}, and {\tt x}, and a {\tt z} array
    for the zomplex case.

\vspace{0.1in}
\noindent Primary routines for the {\tt cholmod\_triplet} object:
% 3
    \begin{itemize}
    \item {\tt cholmod\_allocate\_triplet}: allocate a triplet matrix.
    \item {\tt cholmod\_free\_triplet}: free a triplet matrix.
    \item {\tt cholmod\_triplet\_to\_sparse}: create a sparse matrix copy of a triplet matrix.
    \end{itemize}

\noindent Secondary routines for the {\tt cholmod\_triplet} object:
% 4
    \begin{itemize}
    \item {\tt cholmod\_reallocate\_triplet}: change the number of entries in a triplet matrix.
    \item {\tt cholmod\_sparse\_to\_triplet}: create a triplet matrix copy of a sparse matrix.
    \item {\tt cholmod\_copy\_triplet}: create a copy of a triplet matrix.
    \item {\tt cholmod\_triplet\_xtype}: change the {\tt xtype} of a triplet matrix.
    \end{itemize}

%-------------------------------------------------------------------------------
\subsubsection{Memory management routines}
%-------------------------------------------------------------------------------
    By default, CHOLMOD uses the ANSI C {\tt malloc}, {\tt free},
    {\tt calloc}, and {\tt realloc} routines.  You may use different
    routines by modifying function pointers in the {\tt cholmod\_common} object.

\vspace{0.1in}
\noindent Primary routines:
% 2
    \begin{itemize}
    \item {\tt cholmod\_malloc}: {\tt malloc} wrapper.
    \item {\tt cholmod\_free}: {\tt free} wrapper.
    \end{itemize}

\noindent Secondary routines:
% 3
    \begin{itemize}
    \item {\tt cholmod\_calloc}: {\tt calloc} wrapper.
    \item {\tt cholmod\_realloc}: {\tt realloc} wrapper.
    \item {\tt cholmod\_realloc\_multiple}: {\tt realloc} wrapper for multiple objects.
    \end{itemize}

%-------------------------------------------------------------------------------
\subsubsection{{\tt cholmod\_version:} Version control}
%-------------------------------------------------------------------------------
The {\tt cholmod\_version} function returns the current version of CHOLMOD.

%-------------------------------------------------------------------------------
\newpage \subsection{{\tt Check} Module: print/check the CHOLMOD objects}
%-------------------------------------------------------------------------------
    The {\tt Check} Module contains routines that check and print the five
    basic objects in CHOLMOD, and three kinds of integer vectors (a set,
    a permutation, and a tree).  It also provides a routine to read a sparse
    matrix from a file in Matrix Market format (http://www.nist.gov/MatrixMarket).
    Requires the {\tt Core} Module.

\vspace{0.1in}
\noindent Primary routines:
% 4
    \begin{itemize}
    \item {\tt cholmod\_print\_common}: print the {\tt cholmod\_common} object,
	including statistics on CHOLMOD's behavior (fill-in, flop count,
	ordering methods used, and so on).
    \item {\tt cholmod\_write\_sparse}: write a sparse matrix to a file
	in Matrix Market format.
    \item {\tt cholmod\_write\_dense}: write a sparse matrix to a file
	in Matrix Market format.
    \item {\tt cholmod\_read\_matrix}: read a sparse or dense matrix from a file
	in Matrix Market format.
    \end{itemize}

\vspace{0.1in}
\noindent Secondary routines:
% 18
    \begin{itemize}
    \item {\tt cholmod\_check\_common}: check the {\tt cholmod\_common} object
    \item {\tt cholmod\_check\_sparse}: check a sparse matrix
    \item {\tt cholmod\_print\_sparse}: print a sparse matrix
    \item {\tt cholmod\_check\_dense}: check a dense matrix
    \item {\tt cholmod\_print\_dense}: print a dense matrix
    \item {\tt cholmod\_check\_factor}: check a Cholesky factorization
    \item {\tt cholmod\_print\_factor}: print a Cholesky factorization
    \item {\tt cholmod\_check\_triplet}: check a triplet matrix
    \item {\tt cholmod\_print\_triplet}: print a triplet matrix
    \item {\tt cholmod\_check\_subset}: check a subset (integer vector in given range)
    \item {\tt cholmod\_print\_subset}: print a subset (integer vector in given range)
    \item {\tt cholmod\_check\_perm}: check a permutation (an integer vector)
    \item {\tt cholmod\_print\_perm}: print a permutation (an integer vector)
    \item {\tt cholmod\_check\_parent}: check an elimination tree (an integer vector)
    \item {\tt cholmod\_print\_parent}: print an elimination tree (an integer vector)
    \item {\tt cholmod\_read\_triplet}: read a triplet matrix from a file
    \item {\tt cholmod\_read\_sparse}: read a sparse matrix from a file
    \item {\tt cholmod\_read\_dense}: read a dense matrix from a file
    \end{itemize}

%-------------------------------------------------------------------------------
\newpage \subsection{{\tt Cholesky} Module: sparse Cholesky factorization}
%-------------------------------------------------------------------------------

The primary routines are all that a user requires to order, analyze, and
factorize a sparse symmetric positive definite matrix $\m{A}$ (or $\m{AA}\tr$), and
to solve $\m{Ax}=\m{b}$ (or $\m{AA}\tr\m{x}=\m{b}$).  The primary routines rely on the secondary
routines, the {\tt Core} Module, and the AMD and COLAMD packages.  They
make optional use of the {\tt Supernodal} and {\tt Partition} Modules, the
METIS package, the CAMD package, and
the CCOLAMD package.  The {\tt Cholesky} Module is
required by the {\tt Partition} Module.

\vspace{0.1in}
\noindent Primary routines:
% 4
    \begin{itemize}
    \item {\tt cholmod\_analyze}: order and analyze (simplicial or supernodal).
    \item {\tt cholmod\_factorize}: simplicial or supernodal Cholesky factorization.
    \item {\tt cholmod\_solve}: solve a linear system (simplicial or supernodal, dense $\m{x}$ and $\m{b}$).
    \item {\tt cholmod\_spsolve}: solve a linear system (simplicial or supernodal, sparse $\m{x}$ and $\m{b}$ ).
    \end{itemize}

\noindent Secondary routines:
% 15
    \begin{itemize}
    \item {\tt cholmod\_analyze\_p}: analyze, with user-provided permutation or $\m{f}$ set.
    \item {\tt cholmod\_factorize\_p}: factorize, with user-provided permutation or $\m{f}$.
    \item {\tt cholmod\_analyze\_ordering}:  analyze a permutation
    \item {\tt cholmod\_solve2}: solve a linear system, reusing workspace.
    \item {\tt cholmod\_etree}: find the elimination tree.
    \item {\tt cholmod\_rowcolcounts}: compute the row/column counts of $\m{L}$.
    \item {\tt cholmod\_amd}: order using AMD.
    \item {\tt cholmod\_colamd}: order using COLAMD.
    \item {\tt cholmod\_rowfac}: incremental simplicial factorization.
    \item {\tt cholmod\_row\_subtree}: find the nonzero pattern of a row of $\m{L}$.
    \item {\tt cholmod\_row\_lsubtree}: find the nonzero pattern of a row of $\m{L}$.
    \item {\tt cholmod\_row\_lsubtree}: find the nonzero pattern of $\m{L}^{-1}b$.
    \item {\tt cholmod\_resymbol}: recompute the symbolic pattern of $\m{L}$.
    \item {\tt cholmod\_resymbol\_noperm}: recompute the symbolic pattern of $\m{L}$, no permutation.
    \item {\tt cholmod\_postorder}: postorder a tree.
    \item {\tt cholmod\_rcond}: compute the reciprocal condition number estimate.
    \item {\tt cholmod\_rowfac\_mask}: for use in LPDASA only.
    \end{itemize}

%-------------------------------------------------------------------------------
\newpage \subsection{{\tt Modify} Module: update/downdate a sparse Cholesky factorization}
%-------------------------------------------------------------------------------

The {\tt Modify} Module contains sparse Cholesky modification routines:
update, downdate, row-add, and row-delete.
It can also modify a corresponding solution to $\m{Lx}=\m{b}$ when L is modified.
This module is most useful when applied on a Cholesky factorization computed by
the {\tt Cholesky} module, but it does not actually require the {\tt Cholesky} module.
The {\tt Core} module can create an identity Cholesky factorization ($\m{LDL}\tr$ where
$\m{L}=\m{D}=\m{I}$) that can then be modified by these routines.
Requires the {\tt Core} module.  Not required by any other CHOLMOD Module.

\vspace{0.1in}
\noindent Primary routine:
% 1
    \begin{itemize}
    \item {\tt cholmod\_updown}: multiple rank update/downdate
    \end{itemize}

\noindent Secondary routines:
% 8
    \begin{itemize}
    \item {\tt cholmod\_updown\_solve}: update/downdate, and modify solution to $\m{Lx=b}$
    \item {\tt cholmod\_updown\_mark}: update/downdate, and modify solution to partial $\m{Lx=b}$
    \item {\tt cholmod\_updown\_mask}: for use in LPDASA only.
    \item {\tt cholmod\_rowadd}: add a row to an $\m{LDL}\tr$ factorization
    \item {\tt cholmod\_rowadd\_solve}: add a row, and update solution to $\m{Lx=b}$
    \item {\tt cholmod\_rowadd\_mark}: add a row, and update solution to partial $\m{Lx=b}$
    \item {\tt cholmod\_rowdel}: delete a row from an $\m{LDL}\tr$ factorization
    \item {\tt cholmod\_rowdel\_solve}: delete a row, and downdate $\m{Lx=b}$
    \item {\tt cholmod\_rowdel\_mark}: delete a row, and downdate solution to partial $\m{Lx=b}$
    \end{itemize}

%-------------------------------------------------------------------------------
\subsection{{\tt MatrixOps} Module: basic sparse matrix operations}
%-------------------------------------------------------------------------------

The {\tt MatrixOps} Module provides
basic operations on sparse and dense matrices.
Requires the {\tt Core} module.  Not required by any other CHOLMOD module.
In the descriptions below,
{\tt A}, {\tt B}, and {\tt C:} are sparse matrices ({\tt cholmod\_sparse}),
{\tt X} and {\tt Y} are dense matrices ({\tt cholmod\_dense}),
{\tt s} is a scalar or vector, and
{\tt alpha} {\tt beta} are scalars.

% 10
    \begin{itemize}
    \item {\tt cholmod\_drop}: drop entries from A with absolute value $\ge$ a given tolerance.
    \item {\tt cholmod\_norm\_dense}: {\tt s = norm (X)}, 1-norm, infinity-norm, or 2-norm
    \item {\tt cholmod\_norm\_sparse}: {\tt s = norm (A)}, 1-norm or infinity-norm
    \item {\tt cholmod\_horzcat}: {\tt C = [A,B]}
    \item {\tt cholmod\_scale}: {\tt A = diag(s)*A}, {\tt A*diag(s)}, {\tt s*A} or {\tt diag(s)*A*diag(s)}.
    \item {\tt cholmod\_sdmult}: {\tt Y = alpha*(A*X) + beta*Y} or {\tt alpha*(A'*X) + beta*Y}.
    \item {\tt cholmod\_ssmult}: {\tt C = A*B}
    \item {\tt cholmod\_submatrix}: {\tt C = A (i,j)}, where {\tt i} and {\tt j} are arbitrary integer vectors.
    \item {\tt cholmod\_vertcat}: {\tt C = [A ; B]}.
    \item {\tt cholmod\_symmetry}: determine symmetry of a matrix.
    \end{itemize}

%-------------------------------------------------------------------------------
\newpage \subsection{{\tt Supernodal} Module: supernodal sparse Cholesky factorization}
%-------------------------------------------------------------------------------

The {\tt Supernodal} Module performs
supernodal analysis, factorization, and solve.  The simplest way to use
these routines is via the {\tt Cholesky} Module.  This Module does not provide any
fill-reducing orderings.  It normally operates on matrices ordered by the
{\tt Cholesky} Module.
It does not require the {\tt Cholesky} Module itself, however.
Requires the {\tt Core} Module, and two external packages: LAPACK and the BLAS.
Optionally used by the {\tt Cholesky} Module.  All are secondary routines
since these functions are more easily used via the {\tt Cholesky} Module.

\vspace{0.1in}
\noindent Secondary routines:
% 4
    \begin{itemize}
    \item {\tt cholmod\_super\_symbolic}: supernodal symbolic analysis
    \item {\tt cholmod\_super\_numeric}: supernodal numeric factorization
    \item {\tt cholmod\_super\_lsolve}: supernodal $\m{Lx}=\m{b}$ solve
    \item {\tt cholmod\_super\_ltsolve}: supernodal $\m{L}\tr\m{x}=\m{b}$ solve
    \end{itemize}

%-------------------------------------------------------------------------------
\subsection{{\tt Partition} Module: graph-partitioning-based orderings}
%-------------------------------------------------------------------------------

The {\tt Partition} Module provides
graph partitioning and graph-partition-based orderings.  It includes an
interface to CAMD, CCOLAMD, and CSYMAMD, constrained minimum degree ordering
methods which order a matrix following constraints determined via nested
dissection.
Requires the {\tt Core} and {\tt Cholesky} Modules, and two packages: {\tt METIS 5.1.0}, CAMD, and CCOLAMD.
Optionally used by the {\tt Cholesky} Module.  All are secondary routines since
these are more easily used by the {\tt Cholesky} Module.

Note that METIS does not have a version that uses {\tt long} integers.  If you try to use
these routines (except the CAMD, CCOLAMD, and CSYMAMD interfaces)
on a matrix that is too large, an error code will be returned.

\vspace{0.1in}
\noindent Secondary routines:
% 8
    \begin{itemize}
    \item {\tt cholmod\_nested\_dissection}: CHOLMOD nested dissection ordering
    \item {\tt cholmod\_metis}: METIS nested dissection ordering ({\tt METIS\_NodeND})
    \item {\tt cholmod\_camd}: interface to CAMD ordering
    \item {\tt cholmod\_ccolamd}: interface to CCOLAMD ordering
    \item {\tt cholmod\_csymamd}: interface to CSYMAMD ordering
    \item {\tt cholmod\_bisect}: graph partitioner (currently based on METIS)
    \item {\tt cholmod\_metis\_bisector}: direct interface to {\tt METIS\_NodeComputeSeparator}.
    \item {\tt cholmod\_collapse\_septree}: pruned a separator tree from
    {\tt cholmod\_nested\_dissection}.
    \end{itemize}

%-------------------------------------------------------------------------------
\newpage \section{CHOLMOD naming convention, parameters, and return values}
%-------------------------------------------------------------------------------

All routine names, data types, and CHOLMOD library files use the
{\tt cholmod\_} prefix.  All macros and other {\tt \#define} statements
visible to the user program use the {\tt CHOLMOD} prefix.
The {\tt cholmod.h} file must be included in user programs that use CHOLMOD:

{\footnotesize
\begin{verbatim}
        #include "cholmod.h"
\end{verbatim}
}

\noindent
All CHOLMOD routines (in all modules) use the following protocol for return values:
\begin{itemize}
\item {\tt int}: {\tt TRUE} (1) if successful, or {\tt FALSE} (0) otherwise. (exception: {\tt cholmod\_divcomplex}).
\item {\tt long}: a value $\ge 0$ if successful, or -1 otherwise.
\item {\tt double}: a value $\ge 0$ if successful, or -1 otherwise.
\item {\tt size\_t}: a value $>$ 0 if successful, or 0 otherwise.
\item {\tt void *}: a non-{\tt NULL} pointer to newly allocated memory if successful, or {\tt NULL} otherwise.
\item {\tt cholmod\_sparse *}:  a non-{\tt NULL} pointer to a newly allocated sparse matrix if successful, or {\tt NULL} otherwise.
\item {\tt cholmod\_factor *}:  a non-{\tt NULL} pointer to a newly allocated factor if successful, or {\tt NULL} otherwise.
\item {\tt cholmod\_triplet *}: a non-{\tt NULL} pointer to a newly allocated triplet matrix if successful, or {\tt NULL} otherwise.
\item {\tt cholmod\_dense *}:   a non-{\tt NULL} pointer to a newly allocated dense matrix if successful, or {\tt NULL} otherwise.
\end{itemize}

{\tt TRUE} and {\tt FALSE} are not defined in {\tt cholmod.h},
since they may conflict with the user program.  A routine that described
here returning {\tt TRUE} or {\tt FALSE} returns 1 or 0, respectively.
Any {\tt TRUE}/{\tt FALSE} parameter is true if nonzero, false if zero.

\noindent
Input, output, and input/output parameters:
\begin{itemize}
\item Input parameters appear first in the parameter lists of all CHOLMOD routines.
They are not modified by CHOLMOD.
\item Input/output parameters (except for {\tt Common}) appear next.
They must be defined on input, and are modified on output.
\item Output parameters are listed next.  If they are pointers, they must
point to allocated space on input, but their contents are not defined on input.
\item Workspace parameters appear next.  They are used in only two routines in the Supernodal module.
\item The {\tt cholmod\_common *Common} parameter always appears as the last parameter
(with two exceptions: {\tt cholmod\_hypot} and {\tt cholmod\_divcomplex}).
It is always an input/output parameter.
\end{itemize}

A floating-point scalar is passed to CHOLMOD as a pointer to a {\tt double}
array of size two.  The first entry in this array is the real part of the
scalar, and the second entry is the imaginary part.  The imaginary part is
only accessed if the other inputs are complex or zomplex.  In some cases
the imaginary part is always ignored ({\tt cholmod\_factor\_p}, for example).

%-------------------------------------------------------------------------------
\newpage \section{{\tt Core} Module: {\tt cholmod\_common} object}
\label{cholmod_common}
%-------------------------------------------------------------------------------

%---------------------------------------
\subsection{Constant definitions}
%---------------------------------------

\input{_defn.tex}
These definitions are used within the {\tt cholmod\_common} object,
called {\tt Common} both here and throughout the code.

%---------------------------------------
\newpage \subsection{{\tt cholmod\_common}: parameters, statistics, and workspace}
%---------------------------------------

\input{_common.tex}
The {\tt cholmod\_common Common} object contains parameters, statistics, and
workspace used within CHOLMOD.  The first call to CHOLMOD must be
{\tt cholmod\_start}, which initializes this object.

%---------------------------------------
\newpage \subsection{{\tt cholmod\_start}: start CHOLMOD}
%---------------------------------------

\input{_start.tex}
Sets the default parameters, clears the statistics, and initializes all
workspace pointers to {\tt NULL}.  The {\tt int}/{\tt long} type
is set in {\tt Common->itype}.

%---------------------------------------
\subsection{{\tt cholmod\_finish}: finish CHOLMOD}
%---------------------------------------

\input{_finish.tex}
This must be the last call to CHOLMOD.

%---------------------------------------
\subsection{{\tt cholmod\_defaults}: set default parameters}
%---------------------------------------

\input{_defaults.tex}
Sets the default parameters.

%---------------------------------------
\subsection{{\tt cholmod\_maxrank}: maximum update/downdate rank}
%---------------------------------------

\input{_maxrank.tex}
Returns the maximum rank for an update/downdate.

%---------------------------------------
\subsection{{\tt cholmod\_allocate\_work}: allocate workspace}
%---------------------------------------

\input{_allocate_work.tex}
Allocates workspace in {\tt Common}.  The workspace consists
of the integer {\tt Head}, {\tt Flag}, and {\tt Iwork} arrays,
of size {\tt nrow+1}, {\tt nrow}, and {\tt iworksize},
respectively, and a {\tt double} array {\tt Xwork} of size
{\tt xworksize}.  The {\tt Head} array is normally equal to -1
when it is cleared.  If the {\tt Flag} array is cleared,
all entries are less than {\tt Common->mark}.  The {\tt Iwork} array is
not kept in any particular state.
The integer type is {\tt int} or {\tt long}, depending
on whether the {\tt cholmod\_} or {\tt cholmod\_l\_} routines
are used.

%---------------------------------------
\subsection{{\tt cholmod\_free\_work}: free workspace}
%---------------------------------------

\input{_free_work.tex}
Frees the workspace in {\tt Common}.

%---------------------------------------
\subsection{{\tt cholmod\_clear\_flag}: clear Flag array}
%---------------------------------------

\input{_clear_flag.tex}
Increments {\tt Common->mark} so that the {\tt Flag} array is now cleared.

%---------------------------------------
\newpage \subsection{{\tt cholmod\_error}: report error}
%---------------------------------------

\input{_error.tex}
This routine is called when CHOLMOD encounters an error.
It prints a message (if printing is enabled), sets
{\tt Common->status}.  It then calls the
user error handler routine {\tt Common->error\_handler},
if it is not {\tt NULL}.

%---------------------------------------
\subsection{{\tt cholmod\_dbound}: bound diagonal of $\m{L}$}
%---------------------------------------

\input{_dbound.tex}
Ensures that entries on the diagonal of $\m{L}$ for an $\m{LL}\tr$
factorization are greater than or equal to {\tt Common->dbound}.
For an $\m{LDL}\tr$ factorization, it ensures that the magnitude
of the entries of $\m{D}$ are greater than or equal to {\tt Common->dbound}.

%---------------------------------------
\subsection{{\tt cholmod\_hypot}: {\tt sqrt(x*x+y*y)}}
%---------------------------------------

\input{_hypot.tex}
Computes the magnitude of a complex number.
This routine is the default value for the {\tt Common->hypotenuse} function pointer.
See also {\tt hypot}, in the standard {\tt math.h} header.  If you have
the ANSI C99 {\tt hypot}, you can use {\tt Common->hypotenuse = hypot}.
The {\tt cholmod\_hypot} routine is provided in case you are using the
ANSI C89 standard, which does not have {\tt hypot}.

%---------------------------------------
\subsection{{\tt cholmod\_divcomplex}: complex divide}
%---------------------------------------

\input{_divcomplex.tex}
Divides two complex numbers.  It returns 1 if a divide-by-zero occurred, or 0 otherwise.
This routine is the default value for the {\tt Common->complex\_divide} function pointer.
This return value is the single exception to the CHOLMOD rule that states all {\tt int} return
values are {\tt TRUE} if successful or {\tt FALSE} otherwise.
The exception is made to match the return value of a different complex divide routine
that is not a part of CHOLMOD, but can be used via the function pointer.

%-------------------------------------------------------------------------------
\newpage \section{{\tt Core} Module: {\tt cholmod\_sparse} object}
\label{cholmod_sparse}
%-------------------------------------------------------------------------------

%---------------------------------------
\subsection{{\tt cholmod\_sparse}: compressed-column sparse matrix}
%---------------------------------------

\input{_sparse.tex}
Stores a sparse matrix in compressed-column form.

%---------------------------------------
\subsection{{\tt cholmod\_allocate\_sparse}: allocate sparse matrix}
%---------------------------------------

\input{_allocate_sparse.tex}
Allocates a sparse matrix.  {\tt A->i}, {\tt A->x}, and {\tt A->z} are not initialized.
The matrix returned is all zero, but it contains space enough for {\tt nzmax} entries.

%---------------------------------------
\subsection{{\tt cholmod\_free\_sparse}: free sparse matrix}
%---------------------------------------

\input{_free_sparse.tex}
Frees a sparse matrix.

%---------------------------------------
\subsection{{\tt cholmod\_reallocate\_sparse}: reallocate sparse matrix}
%---------------------------------------

\input{_reallocate_sparse.tex}
Reallocates a sparse matrix, so that it can contain {\tt nznew} entries.

%---------------------------------------
\subsection{{\tt cholmod\_nnz}: number of entries in sparse matrix}
%---------------------------------------

\input{_nnz.tex}
Returns the number of entries in a sparse matrix.

%---------------------------------------
\subsection{{\tt cholmod\_speye}: sparse identity matrix}
%---------------------------------------

\input{_speye.tex}
Returns the sparse identity matrix.

%---------------------------------------
\subsection{{\tt cholmod\_spzeros}: sparse zero matrix}
%---------------------------------------

\input{_spzeros.tex}
Returns the sparse zero matrix.  This is another name
for {\tt cholmod\_allocate\_sparse}, but with fewer parameters
(the matrix is packed, sorted, and unsymmetric).

%---------------------------------------
\newpage \subsection{{\tt cholmod\_transpose}: transpose sparse matrix}
%---------------------------------------

\input{_transpose.tex}
Returns the transpose or complex conjugate transpose of a sparse matrix.

%---------------------------------------
\subsection{{\tt cholmod\_ptranspose}: transpose/permute sparse matrix}
%---------------------------------------

\input{_ptranspose.tex}
Returns {\tt A'} or {\tt A(p,p)'} if {\tt A} is symmetric.
Returns {\tt A'}, {\tt A(:,f)'}, or {\tt A(p,f)'} if {\tt A} is unsymmetric.
See {\tt cholmod\_transpose\_unsym} for a discussion of how {\tt f} is used;
this usage deviates from the MATLAB notation.
Can also return the array transpose.

%---------------------------------------
\subsection{{\tt cholmod\_sort}: sort columns of a sparse matrix}
%---------------------------------------

\input{_sort.tex}
Sorts the columns of the matrix {\tt A}.  Returns  {\tt A} in packed form, even if it
starts as unpacked.  Removes entries in the ignored part of a symmetric matrix.

%---------------------------------------
\newpage \subsection{{\tt cholmod\_transpose\_unsym}: transpose/permute unsymmetric sparse matrix}
%---------------------------------------

\input{_transpose_unsym.tex}
Transposes and optionally permutes an unsymmetric sparse matrix.
The output matrix must be preallocated before calling this routine.

Computes {\tt F=A'}, {\tt F=A(:,f)'} or {\tt F=A(p,f)'}, except that the
indexing by {\tt f} does not work the same as the MATLAB notation (see below).
{\tt A->stype} is zero, which denotes that both the upper and lower triangular
parts of A are present (and used).  The matrix {\tt A} may in fact be symmetric in pattern
and/or value; {\tt A->stype} just denotes which part of {\tt A} are stored.  {\tt A} may be
rectangular.
 
The integer vector
{\tt p} is a permutation of {\tt 0:m-1}, and {\tt f} is a subset of {\tt 0:n-1},
where A is {\tt m}-by-{\tt n}.
There can be no duplicate entries in {\tt p} or {\tt f}.

\noindent
Three kinds of transposes are available, depending on the {\tt values} parameter:
\begin{itemize}
\item    0: do not transpose the numerical values; create a {\tt CHOLMOD\_PATTERN} matrix
\item    1: array transpose
\item    2: complex conjugate transpose (same as 2 if input is real or pattern)
\end{itemize}

\noindent
The set {\tt f} is held in fset and fsize:
\begin{itemize}
\item  {\tt fset = NULL} means ``{\tt :}'' in MATLAB. {\tt fset} is ignored.
\item  {\tt fset != NULL} means {\tt f = fset [0..fsize-1]}.
\item  {\tt fset != NULL} and {\tt fsize = 0} means {\tt f} is the empty set.
\end{itemize}
 
Columns not in the set {\tt f} are considered to be zero.  That is,
if {\tt A} is 5-by-10 then {\tt F=A(:,[3 4])'} is not 2-by-5, but 10-by-5,
and rows 3 and 4 of {\tt F} are equal to columns 3 and 4 of {\tt A} (the other
rows of {\tt F} are zero).  More precisely, in MATLAB notation:

\begin{verbatim}
        [m n] = size (A)
        F = A
        notf = ones (1,n)
        notf (f) = 0
        F (:, find (notf)) = 0
        F = F'
\end{verbatim}

If you want the MATLAB equivalent {\tt F=A(p,f)} operation, use
{\tt cholmod\_submatrix} instead (which does not compute the transpose).
{\tt F->nzmax} must be large enough to hold the matrix {\tt F}.
If {\tt F->nz} is present then {\tt F->nz [j]} is equal to the number of entries in column {\tt j} of {\tt F}.
{\tt A} can be sorted or unsorted, with packed or unpacked columns.
If {\tt f} is present and not sorted in ascending order, then {\tt F} is unsorted
(that is, it may contain columns whose row indices do not appear in
ascending order).  Otherwise, {\tt F} is sorted (the row indices in each
column of {\tt F} appear in strictly ascending order).

{\tt F} is returned in packed or unpacked form, depending on {\tt F->packed} on input.
If {\tt F->packed} is {\tt FALSE}, then {\tt F} is returned in unpacked form ({\tt F->nz} must be
present).  Each row {\tt i} of {\tt F} is large enough to hold all the entries in row {\tt i}
of {\tt A}, even if {\tt f} is provided.  That is, {\tt F->i} and
{\tt F->x [F->p [i] .. F->p [i] + F->nz [i] - 1]} contain all entries in {\tt A(i,f)},
but {\tt F->p [i+1] - F->p [i]} is equal to the number of nonzeros in {\tt A (i,:)},
not just {\tt A (i,f)}.
The {\tt cholmod\_transpose\_unsym} routine is the only operation in CHOLMOD that
can produce an unpacked matrix.

%---------------------------------------
\subsection{{\tt cholmod\_transpose\_sym}: transpose/permute symmetric sparse matrix}
%---------------------------------------

\input{_transpose_sym.tex}
Computes {\tt F = A'} or {\tt F = A(p,p)'}, the transpose or permuted transpose, where
{\tt A->stype} is nonzero.  {\tt A} must be square and symmetric.
If {\tt A->stype} $> 0$, then {\tt A} is a symmetric matrix where just the upper part
of the matrix is stored.  Entries in the lower triangular part may be
present, but are ignored.
If {\tt A->stype} $< 0$, then {\tt A} is a symmetric matrix where just the lower part
of the matrix is stored.  Entries in the upper triangular part may be present, but are ignored.
If {\tt F=A'}, then {\tt F} is returned
sorted; otherwise {\tt F} is unsorted for the {\tt F=A(p,p)'} case.
There can be no duplicate entries in {\tt p}.

Three kinds of transposes are available, depending on the {\tt values} parameter:
\begin{itemize}
\item    0: do not transpose the numerical values; create a {\tt CHOLMOD\_PATTERN} matrix
\item    1: array transpose
\item    2: complex conjugate transpose (same as 2 if input is real or pattern)
\end{itemize}

For {\tt cholmod\_transpose\_unsym} and {\tt cholmod\_transpose\_sym}, the output matrix
{\tt F} must already be pre-allocated by the caller, with the correct dimensions.
If {\tt F} is not valid or has the wrong dimensions, it is not modified.
Otherwise, if {\tt F} is too small, the transpose is not computed; the contents
of {\tt F->p} contain the column pointers of the resulting matrix, where
{\tt F->p [F->ncol] > F->nzmax}.  In this case, the remaining contents of {\tt F} are
not modified.  {\tt F} can still be properly freed with {\tt cholmod\_free\_sparse}.

%---------------------------------------
\subsection{{\tt cholmod\_band}: extract band of a sparse matrix}
%---------------------------------------

\input{_band.tex}
Returns {\tt C = tril (triu (A,k1), k2)}.
{\tt C} is a matrix consisting of the diagonals of A from {\tt k1} to {\tt k2}.
{\tt k=0} is the main diagonal of {\tt A}, {\tt k=1} is the superdiagonal, {\tt k=-1} is the
subdiagonal, and so on.  If {\tt A} is {\tt m}-by-{\tt n}, then:
\begin{itemize}
\item {\tt k1=-m} means {\tt C = tril (A,k2)}
\item {\tt k2=n} means {\tt C = triu (A,k1)}
\item {\tt k1=0} and {\tt k2=0} means {\tt C = diag(A)}, except {\tt C} is a matrix, not a vector
\end{itemize}
Values of {\tt k1} and {\tt k2} less than {\tt -m} are treated as {\tt -m}, and values greater
than {\tt n} are treated as {\tt n}.

{\tt A} can be of any symmetry (upper, lower, or unsymmetric); {\tt C} is returned in
the same form, and packed.  If {\tt A->stype} $> 0$, entries in the lower
triangular part of {\tt A} are ignored, and the opposite is true if
{\tt A->stype} $< 0$.  If {\tt A} has sorted columns, then so does {\tt C}.
{\tt C} has the same size as {\tt A}.

{\tt C} can be returned as a numerical valued matrix (if {\tt A} has numerical values
and {\tt mode} $> 0$), as a pattern-only ({\tt mode} $=0$), or as a pattern-only but with
the diagonal entries removed ({\tt mode} $< 0$).

The xtype of {\tt A} can be pattern or real.  Complex or zomplex cases are supported only
if {\tt mode} is $\le 0$ (in which case the numerical values are ignored).

%---------------------------------------
\subsection{{\tt cholmod\_band\_inplace}: extract band, in place}
%---------------------------------------

\input{_band_inplace.tex}
Same as {\tt cholmod\_band}, except that it always operates in place.
Only packed matrices can be converted in place.

%---------------------------------------
\subsection{{\tt cholmod\_aat}: compute $\m{AA}\tr$}
%---------------------------------------

\input{_aat.tex}
Computes {\tt C = A*A'} or {\tt C = A(:,f)*A(:,f)'}.
{\tt A} can be packed or unpacked, sorted or unsorted, but must be stored with
both upper and lower parts ({\tt A->stype} of zero).  {\tt C} is returned as packed,
{\tt C->stype} of zero (both upper and lower parts present), and unsorted.  See
{\tt cholmod\_ssmult} in the {\tt MatrixOps} Module for a more general matrix-matrix
multiply.
The xtype of {\tt A} can be pattern or real.  Complex or zomplex cases are supported only
if {\tt mode} is $\le 0$ (in which case the numerical values are ignored).
You can trivially convert {\tt C} to a symmetric upper/lower matrix
by changing {\tt C->stype} to 1 or -1, respectively, after calling this routine.

%---------------------------------------
\subsection{{\tt cholmod\_copy\_sparse}: copy sparse matrix}
%---------------------------------------

\input{_copy_sparse.tex}
Returns an exact copy of the input sparse matrix {\tt A}.

%---------------------------------------
\subsection{{\tt cholmod\_copy}: copy (and change) sparse matrix}
%---------------------------------------

\input{_copy.tex}
{\tt C = A}, which allocates {\tt C} and copies {\tt A} into {\tt C}, with possible change of
{\tt stype}.  The diagonal can optionally be removed.  The numerical entries
can optionally be copied.  This routine differs from {\tt cholmod\_copy\_sparse},
which makes an exact copy of a sparse matrix.

{\tt A} can be of any type (packed/unpacked, upper/lower/unsymmetric).  {\tt C} is
packed and can be of any stype (upper/lower/unsymmetric), except that if
{\tt A} is rectangular {\tt C} can only be unsymmetric.  If the stype of A and C
differ, then the appropriate conversion is made.

\noindent
There are three cases for {\tt A->stype}:
\begin{itemize}
\item $<0$,  lower: assume {\tt A} is symmetric with just {\tt tril(A)} stored; the rest of {\tt A} is ignored
\item $ 0$,  unsymmetric: assume {\tt A} is unsymmetric; consider all entries in A
\item $>0$,  upper: assume {\tt A} is symmetric with just {\tt triu(A)} stored; the rest of {\tt A} is ignored
\end{itemize}

\noindent
There are three cases for the requested symmetry of {\tt C} ({\tt stype} parameter):
\begin{itemize}
\item $<0$,  lower: return just {\tt tril(C)}
\item $0$,   unsymmetric: return all of {\tt C}
\item $>0$,  upper: return just {\tt triu(C)}
\end{itemize}

\noindent
This gives a total of nine combinations: \newline
\begin{tabular}{ll}
\hline
Equivalent MATLAB statements			    & Using {\tt cholmod\_copy} \\
\hline
{\tt C = A ;						   }& {\tt A} unsymmetric, {\tt C} unsymmetric \\
{\tt C = tril (A) ;					   }& {\tt A} unsymmetric, {\tt C} lower \\
{\tt C = triu (A) ;					   }& {\tt A} unsymmetric, {\tt C} upper \\
{\tt U = triu (A) ; L = tril (U',-1) ; C = L+U ;	   }& {\tt A} upper, {\tt C} unsymmetric \\
{\tt C = triu (A)' ;					   }& {\tt A} upper, {\tt C} lower \\
{\tt C = triu (A) ;					   }& {\tt A} upper, {\tt C} upper \\
{\tt L = tril (A) ; U = triu (L',1) ; C = L+U ;		   }& {\tt A} lower, {\tt C} unsymmetric \\
{\tt C = tril (A) ;					   }& {\tt A} lower, {\tt C} lower \\
{\tt C = tril (A)' ;					   }& {\tt A} lower, {\tt C} upper \\
\hline
\end{tabular}

\vspace{0.1in}
The xtype of {\tt A} can be pattern or real.  Complex or zomplex cases are supported only
if {\tt values} is {\tt FALSE} (in which case the numerical values are ignored).

%---------------------------------------
\newpage \subsection{{\tt cholmod\_add}: add sparse matrices}
%---------------------------------------

\input{_add.tex}
Returns {\tt C = alpha*A + beta*B}.
If the {\tt stype} of {\tt A} and {\tt B} match, then {\tt C} has
the same {\tt stype}.  Otherwise, {\tt C->stype} is zero ({\tt C} is
unsymmetric).

%---------------------------------------
\subsection{{\tt cholmod\_sparse\_xtype}: change sparse xtype}
%---------------------------------------

\input{_sparse_xtype.tex}
Changes the {\tt xtype} of a sparse matrix, to pattern, real, complex, or zomplex.
Changing from complex or zomplex to real discards the imaginary part.

%-------------------------------------------------------------------------------
\newpage \section{{\tt Core} Module: {\tt cholmod\_factor} object}
\label{cholmod_factor}
%-------------------------------------------------------------------------------

%---------------------------------------
\subsection{{\tt cholmod\_factor} object: a sparse Cholesky factorization}
%---------------------------------------

\input{_factor.tex}
An $\m{LL}\tr$ or $\m{LDL}\tr$ factorization in simplicial or supernodal form.
A simplicial factor is very similar to a {\tt cholmod\_sparse} matrix.
For an $\m{LDL}\tr$ factorization, the diagonal matrix $\m{D}$ is stored
as the diagonal of $\m{L}$; the unit-diagonal of $\m{L}$ is not stored.

%---------------------------------------
\newpage \subsection{{\tt cholmod\_free\_factor}: free factor}
%---------------------------------------

\input{_free_factor.tex}
Frees a factor.

%---------------------------------------
\subsection{{\tt cholmod\_allocate\_factor}: allocate factor}
%---------------------------------------

\input{_allocate_factor.tex}
Allocates a factor and sets it to identity.

%---------------------------------------
\subsection{{\tt cholmod\_reallocate\_factor}: reallocate factor}
%---------------------------------------

\input{_reallocate_factor.tex}
Reallocates a simplicial factor so that it can contain {\tt nznew} entries.

%---------------------------------------
\newpage \subsection{{\tt cholmod\_change\_factor}: change factor}
%---------------------------------------

\input{_change_factor.tex}
Change the numeric or symbolic, $\m{LL}\tr$ or $\m{LDL}\tr$, simplicial or super, packed or unpacked, and
monotonic or non-monotonic status of a {\tt cholmod\_factor} object.

There are four basic classes of factor types:
\begin{enumerate}
\item simplicial symbolic:  Consists of two size-{\tt n} arrays: the fill-reducing
permutation ({\tt L->Perm}) and the nonzero count for each column of L
({\tt L->ColCount}).  All other factor types also include this information.
{\tt L->ColCount} may be exact (obtained from the analysis routines), or
it may be a guess.  During factorization, and certainly after update/downdate,
the columns of {\tt L} can have a different number of nonzeros.
{\tt L->ColCount} is used to allocate space.  {\tt L->ColCount} is exact for the
supernodal factorizations.  The nonzero pattern of {\tt L} is not kept.

\item simplicial numeric:  These represent {\tt L} in a compressed column form.  The
variants of this type are:

\begin{itemize}
\item $\m{LDL}\tr$:	{\tt L} is unit diagonal.  Row indices in column {\tt j} are located in
    {\tt L->i [L->p [j] ... L->p [j] + L->nz [j]]}, and corresponding numeric
    values are in the same locations in {\tt L->x}.  The total number of
    entries is the sum of {\tt L->nz [j]}.  The unit diagonal is not stored;
    {\tt D} is stored on the diagonal of {\tt L} instead.  {\tt L->p} may or may not be
    monotonic.  The order of storage of the columns in {\tt L->i} and {\tt L->x} is
    given by a doubly-linked list ({\tt L->prev} and {\tt L->next}).  {\tt L->p} is of
    size {\tt n+1}, but only the first {\tt n} entries are used.

    For the complex case, {\tt L->x} is stored interleaved with real and imaginary
    parts, and is of size {\tt 2*lnz*sizeof(double)}.  For the zomplex case,
    {\tt L->x} is of size {\tt lnz*sizeof(double)} and holds the real part; {\tt L->z}
    is the same size and holds the imaginary part.

\item $\m{LL}\tr$:  This is identical to the $\m{LDL}\tr$ form, except that the non-unit
    diagonal of {\tt L} is stored as the first entry in each column of {\tt L}.
\end{itemize}

\item supernodal symbolic:  A representation of the nonzero pattern of the
supernodes for a supernodal factorization.  There are {\tt L->nsuper}
supernodes.  Columns {\tt L->super [k]} to {\tt L->super [k+1]-1} are in the {\tt k}th
supernode.  The row indices for the {\tt k}th supernode are in
{\tt L->s [L->pi [k] ... L->pi [k+1]-1]}.  The numerical values are not
allocated ({\tt L->x}), but when they are they will be located in
{\tt L->x [L->px [k] ... L->px [k+1]-1]}, and the {\tt L->px} array is defined
in this factor type.

For the complex case, {\tt L->x} is stored interleaved with real/imaginary parts,
and is of size \newline
{\tt 2*L->xsize*sizeof(double)}.  The zomplex supernodal case
is not supported, since it is not compatible with LAPACK and the BLAS.

\item supernodal numeric:  Always an $\m{LL}\tr$ factorization.  {\tt L} has a non-unit
     diagonal.  {\tt L->x} contains the numerical values of the supernodes, as
     described above for the supernodal symbolic factor.
     For the complex case, {\tt L->x} is stored interleaved, and is of size
     {\tt 2*L->xsize*sizeof(double)}.  The zomplex supernodal case is not
     supported, since it is not compatible with LAPACK and the BLAS.
\end{enumerate}


In all cases, the row indices in each column ({\tt L->i} for simplicial {\tt L} and
{\tt L->s} for supernodal {\tt L}) are kept sorted from low indices to high indices.
This means the diagonal of {\tt L} (or {\tt D} for a $\m{LDL}\tr$ factorization) is always kept as the
first entry in each column.  The elimination tree is not kept.  The parent
of node {\tt j} can be found as the second row index in the {\tt j}th column.
If column {\tt j} has no off-diagonal entries then node {\tt j} is a root
of the elimination tree.

The {\tt cholmod\_change\_factor} routine can do almost all possible conversions.
It cannot do the following conversions:

\begin{itemize}
\item Simplicial numeric types cannot be converted to a supernodal
    symbolic type.  This would simultaneously deallocate the
    simplicial pattern and numeric values and reallocate uninitialized
    space for the supernodal pattern.  This isn't useful for the user,
    and not needed by CHOLMOD's own routines either.
\item Only a symbolic factor (simplicial to supernodal) can be converted
    to a supernodal numeric factor.
\end{itemize}

Some conversions are meant only to be used internally by other CHOLMOD
routines, and should not be performed by the end user.  They allocate space
whose contents are undefined:

\begin{itemize}
\item converting from simplicial symbolic to supernodal symbolic.
\item converting any factor to supernodal numeric.
\end{itemize}

Supports all xtypes, except that there is no supernodal zomplex L.

The {\tt to\_xtype} parameter is used only when converting from symbolic to numeric
or numeric to symbolic.  It cannot be used to convert a numeric xtype (real,
complex, or zomplex) to a different numeric xtype.  For that conversion,
use {\tt cholmod\_factor\_xtype} instead.

%---------------------------------------
\newpage \subsection{{\tt cholmod\_pack\_factor}: pack the columns of a factor}
%---------------------------------------

\input{_pack_factor.tex}
Pack the columns of a simplicial $\m{LDL}\tr$ or $\m{LL}\tr$ factorization.  This can be followed
by a call to {\tt cholmod\_reallocate\_factor} to reduce the size of {\tt L} to the exact
size required by the factor, if desired.  Alternatively, you can leave the
size of {\tt L->i} and {\tt L->x} the same, to allow space for future updates/rowadds.
Each column is reduced in size so that it has at most {\tt Common->grow2} free
space at the end of the column.
Does nothing and returns silently if given any other type of factor.
Does not force the columns of {\tt L} to be monotonic.  It thus differs from
\begin{verbatim}
    cholmod_change_factor (xtype, L->is_ll, FALSE, TRUE, TRUE, L, Common)
\end{verbatim}
which packs the columns and ensures that they appear in monotonic order.

%---------------------------------------
\subsection{{\tt cholmod\_reallocate\_column}: reallocate one column of a factor}
%---------------------------------------

\input{_reallocate_column.tex}
Reallocates the space allotted to a single column of $\m{L}$.

%---------------------------------------
\newpage \subsection{{\tt cholmod\_factor\_to\_sparse}: sparse matrix copy of a factor}
%---------------------------------------

\input{_factor_to_sparse.tex}
Returns a column-oriented sparse matrix containing the pattern and values
of a simplicial or supernodal numerical factor, and then converts the factor
into a simplicial symbolic factor.  If {\tt L} is already packed, monotonic,
and simplicial (which is the case when {\tt cholmod\_factorize} uses the simplicial
Cholesky factorization algorithm) then this routine requires only a small
amount of time and memory, independent of {\tt n}.
It only operates on numeric factors (real, complex, or zomplex).  It does not
change {\tt L->xtype} (the resulting sparse matrix has the same {\tt xtype}
as {\tt L}).  If this routine fails, {\tt L} is left unmodified.

%---------------------------------------
\subsection{{\tt cholmod\_copy\_factor}: copy factor}
%---------------------------------------

\input{_copy_factor.tex}
Returns an exact copy of a factor.

%---------------------------------------
\subsection{{\tt cholmod\_factor\_xtype}: change factor xtype}
%---------------------------------------

\input{_factor_xtype.tex}
Changes the {\tt xtype} of a factor, to pattern, real, complex, or zomplex.
Changing from complex or zomplex to real discards the imaginary part.
You cannot change a supernodal factor to the zomplex xtype.

%-------------------------------------------------------------------------------
\newpage \section{{\tt Core} Module: {\tt cholmod\_dense} object}
\label{cholmod_dense}
%-------------------------------------------------------------------------------

%---------------------------------------
\subsection{{\tt cholmod\_dense} object: a dense matrix}
%---------------------------------------

\input{_dense.tex}
Contains a dense matrix.

%---------------------------------------
\subsection{{\tt cholmod\_allocate\_dense}: allocate dense matrix}
%---------------------------------------

\input{_allocate_dense.tex}
Allocates a dense matrix.

%---------------------------------------
\subsection{{\tt cholmod\_free\_dense}: free dense matrix}
%---------------------------------------

\input{_free_dense.tex}
Frees a dense matrix.

%---------------------------------------
\subsection{{\tt cholmod\_ensure\_dense}: ensure dense matrix has a given size
and type}
%---------------------------------------

\input{_ensure_dense.tex}
Ensures a dense matrix has a given size and type.

%---------------------------------------

%---------------------------------------
\newpage \subsection{{\tt cholmod\_zeros}: dense zero matrix}
%---------------------------------------

\input{_zeros.tex}
Returns an all-zero dense matrix.

%---------------------------------------
\subsection{{\tt cholmod\_ones}: dense matrix, all ones}
%---------------------------------------

\input{_ones.tex}
Returns a dense matrix with each entry equal to one.

%---------------------------------------
\subsection{{\tt cholmod\_eye}: dense identity matrix}
%---------------------------------------

\input{_eye.tex}
Returns a dense identity matrix.

%---------------------------------------
\newpage \subsection{{\tt cholmod\_sparse\_to\_dense}: dense matrix copy of a sparse matrix}
%---------------------------------------

\input{_sparse_to_dense.tex}
Returns a dense copy of a sparse matrix.

%---------------------------------------
\subsection{{\tt cholmod\_dense\_to\_sparse}: sparse matrix copy of a dense matrix}
%---------------------------------------

\input{_dense_to_sparse.tex}
Returns a sparse copy of a dense matrix.

%---------------------------------------
\subsection{{\tt cholmod\_copy\_dense}: copy dense matrix}
%---------------------------------------

\input{_copy_dense.tex}
Returns a copy of a dense matrix.

%---------------------------------------
\newpage \subsection{{\tt cholmod\_copy\_dense2}: copy dense matrix (preallocated)}
%---------------------------------------

\input{_copy_dense2.tex}
Returns a copy of a dense matrix, placing the result in a preallocated matrix {\tt Y}.

%---------------------------------------
\subsection{{\tt cholmod\_dense\_xtype}: change dense matrix xtype}
%---------------------------------------

\input{_dense_xtype.tex}
Changes the {\tt xtype} of a dense matrix, to real, complex, or zomplex.
Changing from complex or zomplex to real discards the imaginary part.

%-------------------------------------------------------------------------------
\newpage \section{{\tt Core} Module: {\tt cholmod\_triplet} object}
\label{cholmod_triplet}
%-------------------------------------------------------------------------------

%---------------------------------------
\subsection{{\tt cholmod\_triplet} object: sparse matrix in triplet form}
%---------------------------------------

\input{_triplet.tex}
Contains a sparse matrix in triplet form.

%---------------------------------------
\subsection{{\tt cholmod\_allocate\_triplet}: allocate triplet matrix}
%---------------------------------------

\input{_allocate_triplet.tex}
Allocates a triplet matrix.

%---------------------------------------
\subsection{{\tt cholmod\_free\_triplet}: free triplet matrix}
%---------------------------------------

\input{_free_triplet.tex}
Frees a triplet matrix.

%---------------------------------------
\newpage \subsection{{\tt cholmod\_reallocate\_triplet}: reallocate triplet matrix}
%---------------------------------------

\input{_reallocate_triplet.tex}
Reallocates a triplet matrix so that it can hold {\tt nznew} entries.

%---------------------------------------
\subsection{{\tt cholmod\_sparse\_to\_triplet}: triplet matrix copy of a sparse matrix}
%---------------------------------------

\input{_sparse_to_triplet.tex}
Returns a triplet matrix copy of a sparse matrix.

%---------------------------------------
\subsection{{\tt cholmod\_triplet\_to\_sparse}: sparse matrix copy of a triplet matrix}
%---------------------------------------

\input{_triplet_to_sparse.tex}
Returns a sparse matrix copy of a triplet matrix.
If the triplet matrix is symmetric with just the lower part present ({\tt T->stype} $< 0$),
then entries in the upper part are transposed and placed in the lower part when
converting to a sparse matrix.  Similarly,
if the triplet matrix is symmetric with just the upper part present ({\tt T->stype} $> 0$),
then entries in the lower part are transposed and placed in the upper part when
converting to a sparse matrix.
Any duplicate entries are summed.

%---------------------------------------
\newpage \subsection{{\tt cholmod\_copy\_triplet}: copy triplet matrix}
%---------------------------------------

\input{_copy_triplet.tex}
Returns an exact copy of a triplet matrix.

%---------------------------------------
\subsection{{\tt cholmod\_triplet\_xtype}: change triplet xtype}
%---------------------------------------

\input{_triplet_xtype.tex}
Changes the {\tt xtype} of a dense matrix, to real, complex, or zomplex.
Changing from complex or zomplex to real discards the imaginary part.

%-------------------------------------------------------------------------------
\newpage \section{{\tt Core} Module: memory management}
%-------------------------------------------------------------------------------

%---------------------------------------
\subsection{{\tt cholmod\_malloc}: allocate memory}
%---------------------------------------

\input{_malloc.tex}
Allocates a block of memory of size {\tt n*size},
using the {\tt SuiteSparse\_config.malloc\_func}
function pointer (default is to use the ANSI C {\tt malloc} routine).
A value of {\tt n=0} is treated as {\tt n=1}.
If not successful, {\tt NULL} is returned and {\tt Common->status} is set to {\tt CHOLMOD\_OUT\_OF\_MEMORY}.

%---------------------------------------
\subsection{{\tt cholmod\_calloc}: allocate and clear memory}
%---------------------------------------

\input{_calloc.tex}
Allocates a block of memory of size {\tt n*size},
using the {\tt SuiteSparse\_config.calloc\_func}
function pointer (default is to use the ANSI C {\tt calloc} routine).
A value of {\tt n=0} is treated as {\tt n=1}.
If not successful, {\tt NULL} is returned and {\tt Common->status} is set to {\tt CHOLMOD\_OUT\_OF\_MEMORY}.

%---------------------------------------
\newpage \subsection{{\tt cholmod\_free}: free memory}
%---------------------------------------

\input{_free.tex}
Frees a block of memory of size {\tt n*size},
using the {\tt SuiteSparse\_config.free\_func}
function pointer (default is to use the ANSI C {\tt free} routine).
The size of the block ({\tt n} and {\tt size}) is only required so that CHOLMOD
can keep track of its current and peak memory usage.  This is a useful statistic,
and it can also help in tracking down memory leaks.  After the call to
{\tt cholmod\_finish}, the count of allocated blocks ({\tt Common->malloc\_count})
should be zero, and the count of bytes in use ({\tt Common->memory\_inuse}) also
should be zero.  If you allocate a block with one size and free it with another,
the {\tt Common->memory\_inuse} count will be wrong, but CHOLMOD will not
have a memory leak.

%---------------------------------------
\subsection{{\tt cholmod\_realloc}: reallocate memory}
%---------------------------------------

\input{_realloc.tex}
Reallocates a block of memory whose current size {\tt n*size},
and whose new size will be {\tt nnew*size} if successful,
using the {\tt SuiteSparse\_config.calloc\_func}
function pointer (default is to use the ANSI C {\tt realloc} routine).
If the reallocation is not successful, {\tt p} is returned unchanged
and {\tt Common->status} is set to {\tt CHOLMOD\_OUT\_OF\_MEMORY}.
The value of {\tt n} is set to {\tt nnew} if successful, or left
unchanged otherwise.
A value of {\tt nnew=0} is treated as {\tt nnew=1}.

%---------------------------------------
\newpage \subsection{{\tt cholmod\_realloc\_multiple}: reallocate memory}
%---------------------------------------

\input{_realloc_multiple.tex}
Reallocates multiple blocks of memory, all with the same number of items
(but with different item sizes).  Either all reallocations succeed,
or all are returned to their original size.

%-------------------------------------------------------------------------------
\newpage \section{{\tt Core} Module: version control}
%-------------------------------------------------------------------------------

\subsection{{\tt cholmod\_version}: return current CHOLMOD version}

\input{_version.tex}
Returns the CHOLMOD version number, so that it can be tested at run time, even
if the caller does not have access to the CHOLMOD include files.  For example,
for a CHOLMOD version 3.2.1, the {\tt version} array will contain 3, 2, and 1,
in that order.  This function appears in CHOLMOD 2.1.1 and later.  You can
check if the function exists with the {\tt CHOLMOD\_HAS\_VERSION\_FUNCTION}
macro, so that the following code fragment works in any version of CHOLMOD:

\begin{verbatim}
   #ifdef CHOLMOD_HAS_VERSION_FUNCTION
   v = cholmod_version (NULL) ;
   #else
   v = CHOLMOD_VERSION ;
   #endif
\end{verbatim}

Note that {\tt cholmod\_version} and {\tt cholmod\_l\_version} have identical
prototypes.  Both use {\tt int}'s.  Unlike all other CHOLMOD functions, this
function does not take the {\tt Common} object as an input parameter, and it
does not use any definitions from any include files.  Thus, the caller can
access this function even if the caller does not include any CHOLMOD include
files.

The above code fragment does require the {\tt \#include "cholmod.h"},
of course, but {\tt cholmod\_version} can be called without it, if necessary.

%-------------------------------------------------------------------------------
\newpage \section{{\tt Check} Module routines}
%-------------------------------------------------------------------------------

No CHOLMOD routines print anything, except for the {\tt cholmod\_print\_*}
routines in the {\tt Check} Module, and the {\tt cholmod\_error} routine.  The
{\tt SuiteSparse\_config.printf\_function} is a pointer to {\tt printf} by default;
you can redirect the output of CHOLMOD by redefining this pointer.
If the function pointer is {\tt NULL}, CHOLMOD does not print anything.

The {\tt Common->print} parameter determines how much detail is printed.
Each value of {\tt Common->print} listed below also prints the items listed
for smaller values of {\tt Common->print}:
\begin{itemize}
\item 0: print nothing; check the data structures and return {\tt TRUE} or {\tt FALSE}.
\item 1: print error messages.
\item 2: print warning messages.
\item 3: print a one-line summary of the object.
\item 4: print a short summary of the object (first and last few entries).
\item 5: print the entire contents of the object.
\end{itemize}
Values less than zero are treated as zero, and values greater than five are
treated as five.

%---------------------------------------
\subsection{{\tt cholmod\_check\_common}: check Common object}
%---------------------------------------

\input{_check_common.tex}
Check if the {\tt Common} object is valid.

%---------------------------------------
\subsection{{\tt cholmod\_print\_common}: print Common object}
%---------------------------------------

\input{_print_common.tex}
Print the {\tt Common} object and check if it is valid.
This prints the CHOLMOD parameters and statistics.

%---------------------------------------
\newpage \subsection{{\tt cholmod\_check\_sparse}: check sparse matrix}
%---------------------------------------

\input{_check_sparse.tex}
Check if a sparse matrix is valid.

%---------------------------------------
\subsection{{\tt cholmod\_print\_sparse}: print sparse matrix}
%---------------------------------------

\input{_print_sparse.tex}
Print a sparse matrix and check if it is valid.

%---------------------------------------
\newpage \subsection{{\tt cholmod\_check\_dense}: check dense matrix}
%---------------------------------------

\input{_check_dense.tex}
Check if a dense matrix is valid.

%---------------------------------------
\subsection{{\tt cholmod\_print\_dense}: print dense matrix}
%---------------------------------------

\input{_print_dense.tex}
Print a dense matrix and check if it is valid.

%---------------------------------------
\newpage \subsection{{\tt cholmod\_check\_factor}: check factor}
%---------------------------------------

\input{_check_factor.tex}
Check if a factor is valid.

%---------------------------------------
\subsection{{\tt cholmod\_print\_factor}: print factor}
%---------------------------------------

\input{_print_factor.tex}
Print a factor and check if it is valid.

%---------------------------------------
\newpage \subsection{{\tt cholmod\_check\_triplet}: check triplet matrix}
%---------------------------------------

\input{_check_triplet.tex}
Check if a triplet matrix is valid.

%---------------------------------------
\subsection{{\tt cholmod\_print\_triplet}: print triplet matrix}
%---------------------------------------

\input{_print_triplet.tex}
Print a triplet matrix and check if it is valid.

%---------------------------------------
\newpage \subsection{{\tt cholmod\_check\_subset}: check subset}
%---------------------------------------

\input{_check_subset.tex}
Check if a subset is valid.

%---------------------------------------
\subsection{{\tt cholmod\_print\_subset}: print subset}
%---------------------------------------

\input{_print_subset.tex}
Print a subset and check if it is valid.

%---------------------------------------
\newpage \subsection{{\tt cholmod\_check\_perm}: check permutation}
%---------------------------------------

\input{_check_perm.tex}
Check if a permutation is valid.

%---------------------------------------
\subsection{{\tt cholmod\_print\_perm}: print permutation}
%---------------------------------------

\input{_print_perm.tex}
Print a permutation and check if it is valid.

%---------------------------------------
\newpage \subsection{{\tt cholmod\_check\_parent}: check elimination tree}
%---------------------------------------

\input{_check_parent.tex}
Check if an elimination tree is valid.

%---------------------------------------
\subsection{{\tt cholmod\_print\_parent}: print elimination tree}
%---------------------------------------

\input{_print_parent.tex}
Print an elimination tree and check if it is valid.

%---------------------------------------
\newpage \subsection{{\tt cholmod\_read\_triplet}: read triplet matrix from file}
%---------------------------------------

\input{_read_triplet.tex}
Read a sparse matrix in triplet form, using the the {\tt coord} 
Matrix Market format (http://www.nist.gov/MatrixMarket).
Skew-symmetric and complex symmetric matrices are returned with
both upper and lower triangular parts present (an stype of zero).
Real symmetric and complex Hermitian matrices are returned with just
their upper or lower triangular part, depending on their stype.
The Matrix Market {\tt array} data type for dense matrices is not supported
(use {\tt cholmod\_read\_dense} for that case).

If the first line of the file starts with {\tt \%\%MatrixMarket}, then it is
interpreted as a file in Matrix Market format.  The header line is optional.
If present, this line must have the following format:
\vspace{0.1in}

        {\tt \%\%MatrixMarket matrix coord} {\em type storage}

\vspace{0.1in}
\noindent
where {\em type} is one of: {\tt real}, {\tt complex}, {\tt pattern},
or {\tt integer}, and {\em storage} is one of: {\tt general}, {\tt hermitian},
{\tt symmetric}, or {\tt skew-symmetric}.
In CHOLMOD, these roughly correspond to the {\tt xtype}
(pattern, real, complex, or zomplex) and {\tt stype}
(unsymmetric, symmetric/upper, and symmetric/lower).
The strings are case-insensitive.  Only the first character (or the
first two for skew-symmetric) is significant.
The {\tt coord} token can be replaced with {\tt array} in the Matrix Market format, but
this format not supported by {\tt cholmod\_read\_triplet}.
The {\tt integer} type is converted to real.
The {\em type} is ignored; the actual type (real, complex, or pattern) is
inferred from the number of tokens in each line of the file (2: pattern,
3: real, 4: complex).  This is compatible with the Matrix Market format.

A storage of {\tt general} implies an stype of zero
(see below).  A storage of {\tt symmetric} and {\tt hermitian} imply an stype of -1.
Skew-symmetric and complex symmetric matrices are returned with an stype of 0.
Blank lines, any other lines starting with ``{\tt \%}'' are treated as comments, and are ignored.

The first non-comment line contains 3 or 4 integers:
\vspace{0.1in}

	{\em nrow ncol nnz stype}

\vspace{0.1in}
\noindent
where {\em stype} is optional (stype does not appear in the Matrix Market format).
The matrix is {\em nrow}-by-{\em ncol}.  The following {\em nnz} lines (excluding comments)
each contain a single entry.  Duplicates are permitted, and are summed in
the output matrix.

If stype is present, it denotes the storage format for the matrix.
\begin{itemize}
\item stype = 0 denotes an unsymmetric matrix (same as Matrix Market {\tt general}).
\item stype = -1 denotes a symmetric or Hermitian matrix whose lower triangular 
	entries are stored.  Entries may be present in the upper triangular
	part, but these are ignored (same as Matrix Market {\tt symmetric}
	for the real case, {\tt hermitian} for the complex case).
\item stype = 1 denotes a symmetric or Hermitian matrix whose upper triangular 
	entries are stored.  Entries may be present in the lower triangular
	part, but these are ignored.  This format is not available in the Matrix
	Market format.
\end{itemize}

If neither the stype nor the Matrix Market header are present, then
the stype is inferred from the rest of the data.  If the matrix is
rectangular, or has
entries in both the upper and lower triangular parts, then it is assumed to
be unsymmetric (stype=0).  If only entries in the lower triangular part are
present, the matrix is assumed to have stype = -1.  If only entries in the
upper triangular part are present, the matrix is assumed to have stype = 1.

Each nonzero consists of one line with 2, 3, or 4 entries.  All lines must
have the same number of entries.  The first two entries are the row and
column indices of the nonzero.  If 3 entries are present, the 3rd entry is
the numerical value, and the matrix is real.  If 4 entries are present,
the 3rd and 4th entries in the line are the real and imaginary parts of
a complex value.

The matrix can be either 0-based or 1-based.  It is first assumed to be
one-based (compatible with Matrix Market), with row indices in the range
1 to ncol and column indices in the range 1 to nrow.  If a row or column
index of zero is found, the matrix is assumed to be zero-based (with row
indices in the range 0 to ncol-1 and column indices in the range 0 to
nrow-1).  This test correctly determines that all Matrix Market
matrices are in 1-based form.

For symmetric pattern-only matrices, the kth diagonal (if present) is set to
one plus the degree of the row k or column k (whichever is larger), and the
off-diagonals are set to -1.  A symmetric pattern-only matrix with a
zero-free diagonal is thus converted into a symmetric positive definite
matrix.  All entries are set to one for an unsymmetric pattern-only matrix.
This differs from the MatrixMarket format ({\tt A = mmread ('file')} returns
a binary pattern for A for symmetric pattern-only matrices).
To return a binary format for all pattern-only matrices, use
{\tt A = mread('file',1)}.

Example matrices that follow this format can be found in the
{\tt CHOLMOD/Demo/Matrix} and \newline
{\tt CHOLMOD/Tcov/Matrix} directories.
You can also try any of the matrices in the Matrix Market collection
at http://www.nist.gov/MatrixMarket.

%---------------------------------------
\subsection{{\tt cholmod\_read\_sparse}: read sparse matrix from file}
%---------------------------------------

\input{_read_sparse.tex}
Read a sparse matrix in triplet form from a file (using {\tt cholmod\_read\_triplet})
and convert to a CHOLMOD sparse matrix.
The Matrix Market format is used.
If {\tt Common->prefer\_upper} is {\tt TRUE} (the default case), a symmetric matrix is
returned stored in upper-triangular form ({\tt A->stype} is 1).
Otherwise, it is left in its original form, either upper or lower.

%---------------------------------------
\newpage \subsection{{\tt cholmod\_read\_dense}: read dense matrix from file}
%---------------------------------------

\input{_read_dense.tex}
Read a dense matrix from a file, using the the {\tt array} 
Matrix Market format 
\newline (http://www.nist.gov/MatrixMarket).

%---------------------------------------
\subsection{{\tt cholmod\_read\_matrix}: read a matrix from file}
%---------------------------------------

\input{_read_matrix.tex}
Read a sparse or dense matrix from a file, in Matrix Market format.
Returns a {\tt void} pointer to either a
{\tt cholmod\_triplet},
{\tt cholmod\_sparse}, or
{\tt cholmod\_dense} object.

%---------------------------------------
\newpage \subsection{{\tt cholmod\_write\_sparse}: write a sparse matrix to a file}
%---------------------------------------

\input{_write_sparse.tex}
Write a sparse matrix to a file in Matrix Market format.   Optionally include
comments, and print explicit zero entries given by the pattern of the {\tt Z}
matrix.  If not NULL, the {\tt Z} matrix must have the same dimensions and stype
as {\tt A}.

Returns the symmetry in which the matrix was printed
(1 to 7) or -1 on failure.  See the {\tt cholmod\_symmetry} function for
a description of the return codes.

If {\tt A} and {\tt Z} are sorted on input, and either unsymmetric (stype = 0) or
symmetric-lower (stype < 0), and if {\tt A} and {\tt Z} do not overlap,
then the triplets
are sorted, first by column and then by row index within each column, with
no duplicate entries.  If all the above holds except stype > 0, then the
triplets are sorted by row first and then column.

%---------------------------------------
\subsection{{\tt cholmod\_write\_dense}: write a dense matrix to a file}
%---------------------------------------

\input{_write_dense.tex}
Write a dense matrix to a file in Matrix Market format.   Optionally include
comments.  Returns > 0 if successful, -1 otherwise (1 if rectangular, 2 if
square).
A dense matrix is written in "general" format; symmetric formats in the
Matrix Market standard are not exploited.

%-------------------------------------------------------------------------------
\newpage \section{{\tt Cholesky} Module routines}
%-------------------------------------------------------------------------------

%---------------------------------------
\subsection{{\tt cholmod\_analyze}: symbolic factorization}
%---------------------------------------

\input{_analyze.tex}
Orders and analyzes a matrix (either simplicial or supernodal), in preparation
for numerical factorization via {\tt cholmod\_factorize} or via the ``expert''
routines {\tt cholmod\_rowfac} and {\tt cholmod\_super\_numeric}.

In the symmetric case, {\tt A} or {\tt A(p,p)} is analyzed,
where {\tt p} is the fill-reducing ordering.
In the unsymmetric case, {\tt A*A'} or {\tt A(p,:)*A(p,:)'} is analyzed.
The {\tt cholmod\_analyze\_p} routine can be given a user-provided permutation {\tt p}
(see below).

The default ordering strategy is to first try AMD.
The ordering quality is analyzed, and if AMD obtains an ordering where
{\tt nnz(L)} is greater than or equal to {\tt 5*nnz(tril(A))}
(or {\tt 5*nnz(tril(A*A'))} if {\tt A} is unsymmetric) and
the floating-point operation count for the subsequent factorization is
greater than or equal to {\tt 500*nnz(L)}, then METIS is tried (if installed).
For {\tt cholmod\_analyze\_p}, the user-provided ordering is also tried.
This default behavior is obtained when {\tt Common->nmethods} is zero.
In this case, methods 0, 1, and 2 in {\tt Common->method[...]} are reset
to user-provided, AMD, and METIS, respectively.
The ordering with the smallest {\tt nnz(L)} is kept.

If {\tt Common->default\_nesdis} is true (nonzero), then CHOLMOD's
nested dissection (NESDIS) is used for the default strategy described
above, in place of METIS.

Other ordering options can be requested.  These include:
\begin{enumerate}
\item	natural:    A is not permuted to reduce fill-in.
\item	user-provided:	    a permutation can be provided to {\tt cholmod\_analyze\_p}.
\item	AMD:	    approximate minimum degree (AMD for the symmetric case, COLAMD for the {\tt A*A'} case).
\item	METIS:	    nested dissection with {\tt METIS\_NodeND}
\item	{\tt NESDIS}:	    CHOLMOD's nested dissection using
    {\tt METIS\_NodeComputeSeparator},
    followed by a constrained minimum degree
    (CAMD or CSYMAMD for the symmetric case, CCOLAMD for the {\tt A*A'} case).
    This is typically slower than METIS, but typically provides better orderings.
\end{enumerate}

Multiple ordering options can be tried (up to 9 of them), and the best one
is selected (the one that gives the smallest number of nonzeros in the
simplicial factor L).  If one method fails, {\tt cholmod\_analyze} keeps going, and
picks the best among the methods that succeeded.  This routine fails (and
returns {\tt NULL}) if either the initial memory allocation fails, all ordering methods
fail, or the supernodal analysis (if requested) fails.  Change {\tt Common->nmethods} to the
number of methods you wish to try.  By default, the 9 methods available are:

\begin{enumerate}
\item user-provided permutation (only for {\tt cholmod\_analyze\_p}).
\item AMD with default parameters.
\item METIS with default parameters.
\item {\tt NESDIS} with default parameters: stopping the partitioning when
	    the graph is of size {\tt nd\_small} = 200 or less, remove nodes with
	    more than {\tt max (16, prune\_dense * sqrt (n))} nodes where
	    {\tt prune\_dense} = 10, and follow partitioning with
	    constrained minimum degree ordering
	    (CAMD for the symmetric case,
	    CCOLAMD for the unsymmetric case).
\item natural ordering (with weighted postorder).
\item NESDIS, {\tt nd\_small} = 20000, {\tt prune\_dense} = 10.
\item NESDIS, {\tt nd\_small} =     4, {\tt prune\_dense} = 10,
    no constrained minimum degree.
\item NESDIS, {\tt nd\_small} =   200, {\tt prune\_dense} = 0.
\item COLAMD for {\tt A*A'} or AMD for {\tt A}
\end{enumerate}

You can modify these 9 methods and the number of methods tried by changing
parameters in the {\tt Common} argument.  If you know the best ordering for your
matrix, set {\tt Common->nmethods} to 1 and set {\tt Common->method[0].ordering} to the
requested ordering method.  Parameters for each method can also be modified
(refer to the description of {\tt cholmod\_common} for details).

Note that it is possible for METIS to terminate your program if it runs out
of memory.  This is not the case for any CHOLMOD or minimum degree ordering
routine (AMD, COLAMD, CAMD, CCOLAMD, or CSYMAMD).
Since {\tt NESDIS} relies on METIS, it too can terminate your program.

The selected ordering is followed by a weighted postorder of the elimination
tree by default (see {\tt cholmod\_postorder} for details),
unless {\tt Common->postorder} is set to {\tt FALSE}.
The postorder does not change the number of nonzeros in $\m{L}$ or
the floating-point operation count.  It does improve performance,
particularly for the supernodal factorization.
If you truly want the natural ordering with no postordering,
you must set {\tt Common->postorder} to {\tt FALSE}.

The factor {\tt L} is returned as simplicial symbolic if
{\tt Common->supernodal} is {\tt CHOLMOD\_SIMPLICIAL} (zero) or as supernodal symbolic if
{\tt Common->supernodal} is {\tt CHOLMOD\_SUPERNODAL} (two).  If \newline
{\tt Common->supernodal} is {\tt CHOLMOD\_AUTO} (one),
then {\tt L} is simplicial if the flop count per nonzero in {\tt L} is less than
{\tt Common->supernodal\_switch} (default: 40), and
supernodal otherwise.  In both cases, {\tt L->xtype} is {\tt CHOLMOD\_PATTERN}.
A subsequent call to {\tt cholmod\_factorize} will perform a
simplicial or supernodal factorization, depending on the type of {\tt L}.

For the simplicial case, {\tt L} contains the fill-reducing permutation ({\tt L->Perm})
and the counts of nonzeros in each column of {\tt L} ({\tt L->ColCount}).  For the
supernodal case, {\tt L} also contains the nonzero pattern of each supernode.

If a simplicial factorization is selected, it will be $\m{LDL}\tr$ by default, since
this is the kind required by the {\tt Modify} Module.  CHOLMOD does not include a
supernodal $\m{LDL}\tr$ factorization, so if a supernodal factorization is selected,
it will be in the form $\m{LL}\tr$.  The $\m{LDL}\tr$ method can be used to
factorize positive definite matrices and indefinite matrices whose leading minors
are well-conditioned (2-by-2 pivoting is not supported).  The $\m{LL}\tr$ method
is restricted to positive definite matrices.  To factorize a large indefinite matrix,
set {\tt Common->supernodal} to {\tt CHOLMOD\_SIMPLICIAL}, and the simplicial 
$\m{LDL}\tr$ method will always be used.  This will be significantly slower than
a supernodal $\m{LL}\tr$ factorization, however.

Refer to {\tt cholmod\_transpose\_unsym} for a description of {\tt f}.

%---------------------------------------
\subsection{{\tt cholmod\_factorize}: numeric factorization}
%---------------------------------------

\input{_factorize.tex}
Computes the numerical factorization of a symmetric matrix.  The % primary
inputs to this routine are a sparse matrix {\tt A} and the symbolic factor {\tt L} from
{\tt cholmod\_analyze} or a prior numerical factor {\tt L}.  If {\tt A} is symmetric, this
routine factorizes {\tt A(p,p)}. %+beta*I (beta can be zero),
where p is the fill-reducing permutation ({\tt L->Perm}).  If {\tt A} is unsymmetric,
% either
{\tt A(p,:)*A(p,:)'} % +beta*I  or A(p,f)*A(p,f)'+beta*I
is factorized.  %  The set f and
The nonzero pattern of the matrix {\tt A} must be the same as the matrix passed to
{\tt cholmod\_analyze} for the supernodal case.  For the simplicial case, it can
be different, but it should be the same for best performance. %  beta is real.

A simplicial factorization or supernodal factorization is chosen, based on
the type of the factor {\tt L}.  If {\tt L->is\_super} is {\tt TRUE}, a supernodal $\m{LL}\tr$
factorization is computed.  Otherwise, a simplicial numeric factorization
is computed, either $\m{LL}\tr$ or $\m{LDL}\tr$, depending on {\tt Common->final\_ll}
(the default for the simplicial case is to compute an $\m{LDL}\tr$ factorization).

Once the factorization is complete, it can be left as is or optionally
converted into any simplicial numeric type, depending on the
{\tt Common->final\_*} parameters.  If converted from a supernodal to simplicial
type, and {\tt Common->final\_resymbol} is {\tt TRUE}, then numerically
zero entries in {\tt L} due to relaxed supernodal amalgamation are removed from
the simplicial factor (they are always left in the supernodal form of {\tt L}).
Entries that are numerically zero but present in the simplicial symbolic
pattern of {\tt L} are left in place (the graph of {\tt L} remains chordal).
This is required for the update/downdate/rowadd/rowdel routines to work
properly.

If the matrix is not positive definite the routine returns {\tt TRUE}, but
{\tt Common->status} is set to {\tt CHOLMOD\_NOT\_POSDEF} and {\tt L->minor} is set to the
column at which the failure occurred.  Columns {\tt L->minor} to {\tt L->n-1}
are set to zero.

Supports any xtype (pattern, real, complex, or zomplex), except that the
input matrix {\tt A} cannot be pattern-only.  If {\tt L} is simplicial, its numeric
xtype matches {\tt A} on output.  If {\tt L} is supernodal, its xtype is real if {\tt A} is
real, or complex if {\tt A} is complex or zomplex.  CHOLMOD does not provide
a supernodal zomplex factor, since it is incompatible with how complex numbers are
stored in LAPACK and the BLAS.

%---------------------------------------
\subsection{{\tt cholmod\_analyze\_p}: symbolic factorization, given permutation}
%---------------------------------------

\input{_analyze_p.tex}
Identical to {\tt cholmod\_analyze}, except that a user-provided
permutation {\tt p} can be provided, and the set {\tt f} for the unsymmetric case
can be provided.  The matrices {\tt A(:,f)*A(:,f)'} or {\tt A(p,f)*A(p,f)'}
can be analyzed in the the unsymmetric case.

%---------------------------------------
\subsection{{\tt cholmod\_factorize\_p}: numeric factorization, given permutation}
%---------------------------------------

\input{_factorize_p.tex}
Identical to {\tt cholmod\_factorize}, but with additional options.
The set {\tt f} can be provided for the unsymmetric case;
{\tt A(p,f)*A(p,f)'} is factorized.  The term {\tt beta*I} can be added to
the matrix before it is factorized, where {\tt beta} is real.
Only the real part, {\tt beta[0]}, is used.

%---------------------------------------
\subsection{{\tt cholmod\_solve}: solve a linear system}
%---------------------------------------

\input{_solve.tex}
Returns a solution {\tt X} that solves one of the following systems:

\begin{tabular}{ll|ll}
	\hline
	system			    & {\tt sys} parameter & system		    & {\tt sys} parameter \\
	$\m{Ax}=\m{b}$		    & 0: {\tt CHOLMOD\_A}	&			    &			\\
	$\m{LDL}\tr\m{x}=\m{b}$	    & 1: {\tt CHOLMOD\_LDLt}	& $\m{L}\tr\m{x}=\m{b}$	    & 5: {\tt CHOLMOD\_Lt}	\\
	$\m{LDx}=\m{b}$		    & 2: {\tt CHOLMOD\_LD}	& $\m{Dx}=\m{b}$	    & 6: {\tt CHOLMOD\_D}	\\
	$\m{DL}\tr\m{x}=\m{b}$	    & 3: {\tt CHOLMOD\_DLt}	& $\m{x}=\m{Pb}$	    & 7: {\tt CHOLMOD\_P}	\\
	$\m{Lx}=\m{b}$		    & 4: {\tt CHOLMOD\_L}	& $\m{x}=\m{P}\tr\m{b}$	    & 8: {\tt CHOLMOD\_Pt}	\\
	\hline
\end{tabular}

The factorization can be simplicial $\m{LDL}\tr$, simplicial $\m{LL}\tr$, or supernodal $\m{LL}\tr$.
For an $\m{LL}\tr$ factorization, $\m{D}$ is the identity matrix.  Thus {\tt CHOLMOD\_LD} and
{\tt CHOLMOD\_L} solve the same system if an $\m{LL}\tr$ factorization was performed,
for example.
This is one of the few routines in CHOLMOD for which the xtype of the input
arguments need not match.
If both {\tt L} and {\tt B} are real, then {\tt X} is returned real.  If either is complex
or zomplex, {\tt X} is returned as either complex or zomplex, depending on the
{\tt Common->prefer\_zomplex} parameter (default is complex).

This routine does not check to see if the diagonal of $\m{L}$ or $\m{D}$ is zero,
because sometimes a partial solve can be done with an indefinite or singular
matrix.  If you wish to check in your own code, test {\tt L->minor}.  If
{\tt L->minor == L->n}, then the matrix has no zero diagonal entries.
If {\tt k = L->minor < L->n}, then {\tt L(k,k)} is zero for an $\m{LL}\tr$ factorization, or
{\tt D(k,k)} is zero for an $\m{LDL}\tr$ factorization.

Iterative refinement is not performed, but this can be easily done with
the {\tt MatrixOps} Module.  See {\tt Demo/cholmod\_demo.c} for an example.

%---------------------------------------
\subsection{{\tt cholmod\_spsolve}: solve a linear system}
%---------------------------------------

\input{_spsolve.tex}
Identical to {\tt cholmod\_solve}, except that {\tt B} and {\tt X} are sparse.
This function converts {\tt B} to full format, solves the system, and then
converts {\tt X} back to sparse.  If you want to solve with a sparse {\tt B}
and get just a partial solution back in {\tt X} (corresponding to the pattern
of {\tt B}), use {\tt cholmod\_solve2} below.

%---------------------------------------
\subsection{{\tt cholmod\_solve2}: solve a linear system, reusing workspace}
%---------------------------------------

\input{_solve2.tex}
Solve a linear system, optionally reusing workspace from a prior call
to {\tt cholmod\_solve2}.

The inputs to this function are the same as {\tt cholmod\_solve},
with the addition of three parameters:
{\tt X},
{\tt Y},
and
{\tt E}.  The dense matrix {\tt X} is the solution on output.
On input, {\tt \&X} can point to a NULL matrix, or be the wrong size.
If that is the case, it is freed and allocated to be the proper size. 
If {\tt X} has the right size and type on input, then the allocation
is skipped.  In contrast, the {\tt cholmod\_solve} function always
allocates its output {\tt X}.  This {\tt cholmod\_solve2} function
allows you to reuse the memory space of a prior {\tt X}, thereby
saving time.

The two workspace matrices {\tt Y} and {\tt E} can also be reused
between calls.
You must free
{\tt X}
{\tt Y},
and 
{\tt E} yourself, when your computations are done.  Below is an
example of usage.
Note that 
{\tt X}
{\tt Y},
and 
{\tt E} must be defined on input (either NULL, or valid dense matrices).

\begin{verbatim}
        cholmod_dense *X = NULL, *Y = NULL, *E = NULL ;
        ...
        cholmod_l_solve2 (sys, L, B1, NULL, &X, NULL, &Y, &E, Common) ;
        cholmod_l_solve2 (sys, L, B2, NULL, &X, NULL, &Y, &E, Common) ;
        cholmod_l_solve2 (sys, L, B3, NULL, &X, NULL, &Y, &E, Common) ;
        cholmod_l_free_dense (&X, Common) ;
        cholmod_l_free_dense (&Y, Common) ;
        cholmod_l_free_dense (&E, Common) ;
\end{verbatim}

The equivalent when using {\tt cholmod\_solve} is:

\begin{verbatim}
        cholmod_dense *X = NULL, *Y = NULL, *E = NULL ;
        ...
        X = cholmod_l_solve (sys, L, B1, Common) ;
        cholmod_l_free_dense (&X, Common) ;
        X = cholmod_l_solve (sys, L, B2, Common) ;
        cholmod_l_free_dense (&X, Common) ;
        X = cholmod_l_solve (sys, L, B3, Common) ;
        cholmod_l_free_dense (&X, Common) ;
\end{verbatim}

Both methods work fine, but in the second method with {\tt cholmod\_solve},
the internal workspaces ({\tt Y} and {\tt E}) and the solution ({\tt X})
are allocated and freed on each call.

The {\tt cholmod\_solve2 function} can also solve for a subset of the solution
vector {\tt X}, if the optional {\tt Bset} parameter is non-NULL.  The
right-hand-side {\tt B} must be a single column vector, and its complexity
(real, complex, zomplex) must match that of {\tt L}.  The vector {\tt B} is
dense, but it is assumed to be zero except for row indices specified in {\tt
Bset}.  The vector {\tt Bset} must be a sparse column vector, of dimension the
same as {\tt B}.  Only the pattern of {\tt Bset} is used.  The solution {\tt X}
(a dense column vector) is modified on output, but is defined only in the rows
defined by the sparse vector {\tt Xset}.  The entries in {\tt Bset} are a
subset of {\tt Xset} (except if {\tt sys} is {\tt CHOLMOD\_P} or {\tt
CHOLMOD\_Pt}).

No memory allocations are done if the outputs and internal
workspaces ({\tt X}, {\tt Xset}, {\tt Y}, and {\tt E}) have been allocated
by a prior call (or if allocated by the user).  To let {\tt cholmod\_solve2}
allocate these outputs and workspaces for you, simply initialize them to
NULL (as in the example above).  Since it is possible for this function
to reallocate these 4 arrays, you should always re-acquire the pointers to
their internal data ({\tt X->x} for example) after calling
{\tt cholmod\_solve2}), since they may change.  They normally will not
change except in the first call to this function.

On the first call to {\tt cholmod\_solve2} when {\tt Bset} is NULL,
the factorization is converted from supernodal to simplicial, if needed.
The inverse permutation is also computed and stored in the factorization
object, {\tt L}.  This can take a modest amount of time.  Subsequent
calls to {\tt cholmod\_solve2} with a small {\tt Bset}
are very fast (both asymptotically and in practice).

You can find an example of how to use {\tt cholmod\_solve2} in the
two demo programs, {\tt cholmod\_demo} and {\tt cholmod\_l\_demo}.

%---------------------------------------
\subsection{{\tt cholmod\_etree}: find elimination tree}
%---------------------------------------

\input{_etree.tex}
Computes the elimination tree of {\tt A} or {\tt A'*A}.
In the symmetric case, the upper triangular part of {\tt A} is used.  Entries not
in this part of the matrix are ignored.  Computing the etree of a symmetric
matrix from just its lower triangular entries is not supported.
In the unsymmetric case, all of {\tt A} is used, and the etree of {\tt A'*A} is computed.
Refer to \cite{Liu90a} for a discussion of the elimination tree
and its use in sparse Cholesky factorization.
% J. Liu, "The role of elimination trees in sparse factorization", SIAM J.
% Matrix Analysis \& Applic., vol 11, 1990, pp. 134-172.

%---------------------------------------
\subsection{{\tt cholmod\_rowcolcounts}: nonzeros counts of a factor}
%---------------------------------------

\input{_rowcolcounts.tex}
Compute the row and column counts of the Cholesky factor {\tt L} of the matrix
{\tt A} or {\tt A*A'}.  The etree and its postordering must already be computed (see
{\tt cholmod\_etree} and {\tt cholmod\_postorder}) and given as inputs to this routine.
For the symmetric case ($\m{LL}\tr=\m{A}$), {\tt A} must be stored in
symmetric/lower form ({\tt A->stype = -1}).
In the unsymmetric case, {\tt A*A'} or {\tt A(:,f)*A(:,f)'} can be analyzed.
The fundamental floating-point operation count is returned in {\tt Common->fl}
(this excludes extra flops due to relaxed supernodal amalgamation).
Refer to {\tt cholmod\_transpose\_unsym} for a description of {\tt f}.
The algorithm is described in \cite{GilbertLiNgPeyton01,GilbertNgPeyton94}.

% J. Gilbert, E. Ng, B. Peyton, "An efficient algorithm to compute row and
% column counts for sparse Cholesky factorization", SIAM J. Matrix Analysis \&
% Applic., vol 15, 1994, pp. 1075-1091.
% 
% J. Gilbert, X. Li, E. Ng, B. Peyton, "Computing row and column counts for
% sparse QR and LU factorization", BIT, vol 41, 2001, pp. 693-710.

%---------------------------------------
\subsection{{\tt cholmod\_analyze\_ordering}: analyze a permutation}
%---------------------------------------

\input{_analyze_ordering.tex}
Given a matrix {\tt A} and its fill-reducing permutation, compute the elimination
tree, its (non-weighted) postordering, and the number of nonzeros in each
column of {\tt L}.  Also computes the flop count, the total nonzeros in {\tt L}, and
the nonzeros in {\tt tril(A)} ({\tt Common->fl}, {\tt Common->lnz}, and {\tt Common->anz}).
In the unsymmetric case, {\tt A(p,f)*A(p,f)'} is analyzed, and {\tt Common->anz}
is the number of nonzero entries in the lower triangular part of the product,
not in {\tt A} itself.

Refer to {\tt cholmod\_transpose\_unsym} for a description of {\tt f}.

The column counts of {\tt L}, flop count, and other statistics from
{\tt cholmod\_rowcolcounts} are not computed if {\tt ColCount} is {\tt NULL}.

%---------------------------------------
\subsection{{\tt cholmod\_amd}: interface to AMD}
%---------------------------------------

\input{_amd.tex}
CHOLMOD interface to the AMD ordering package.  Orders {\tt A} if the matrix is
symmetric.  On output, {\tt Perm [k] = i} if row/column {\tt i} of {\tt A} is the {\tt k}th
row/column of {\tt P*A*P'}.  This corresponds to {\tt A(p,p)} in MATLAB notation.
If A is unsymmetric, {\tt cholmod\_amd} orders {\tt A*A'} or {\tt A(:,f)*A(:,f)'}.
On output, {\tt Perm [k] = i} if
row/column {\tt i} of {\tt A*A'} is the {\tt k}th row/column of {\tt P*A*A'*P'}.
This corresponds to {\tt A(p,:)*A(p,:)'} in MATLAB notation.
If {\tt f} is present, {\tt A(p,f)*A(p,f)'} is the permuted matrix.
Refer to {\tt cholmod\_transpose\_unsym} for a description of {\tt f}.

Computes the flop count for a subsequent $\m{LL}\tr$ factorization, the number
of nonzeros in {\tt L}, and the number of nonzeros in the matrix ordered
({\tt A}, {\tt A*A'} or {\tt A(:,f)*A(:,f)'}).
These statistics are returned in
{\tt Common->fl}, {\tt Common->lnz}, and {\tt Common->anz}, respectively.

%---------------------------------------
\subsection{{\tt cholmod\_colamd}: interface to COLAMD}
%---------------------------------------

\input{_colamd.tex}
CHOLMOD interface to the COLAMD ordering package.
Finds a permutation {\tt p} such that the Cholesky factorization of {\tt P*A*A'*P'} is
sparser than {\tt A*A'}, using COLAMD.  If the {\tt postorder} input parameter is {\tt TRUE},
the column elimination tree is found and postordered, and the COLAMD ordering is then
combined with its postordering (COLAMD itself does not perform this postordering).
{\tt A} must be unsymmetric ({\tt A->stype = 0}).

%---------------------------------------
\subsection{{\tt cholmod\_rowfac}: row-oriented Cholesky factorization}
%---------------------------------------

\input{_rowfac.tex}
Full or incremental numerical $\m{LDL}\tr$ or $\m{LL}\tr$ factorization (simplicial, not
supernodal).  {\tt cholmod\_factorize} is the ``easy'' wrapper for this code, but it
does not provide access to incremental factorization.
The algorithm is the row-oriented, up-looking method described in \cite{Davis05}.
See also \cite{Liu86c}.  No 2-by-2 pivoting (or any other pivoting) is performed.

{\tt cholmod\_rowfac} computes the full or incremental $\m{LDL}\tr$ or $\m{LL}\tr$ factorization of
{\tt A+beta*I} (where {\tt A} is symmetric) or {\tt A*F+beta*I}
(where {\tt A} and {\tt F} are unsymmetric
and only the upper triangular part of {\tt A*F+beta*I} is used).  It computes
{\tt L} (and {\tt D}, for $\m{LDL}\tr$) one row at a time.
The input scalar {\tt beta} is real; only the real part ({\tt beta[0]}) is used.

{\tt L} can be a simplicial symbolic or numeric ({\tt L->is\_super} must be {\tt FALSE}).
A symbolic factor is converted immediately into a numeric factor containing
the identity matrix.

For a full factorization, use {\tt kstart = 0} and {\tt kend = nrow}.  The existing nonzero
entries (numerical values in {\tt L->x} and {\tt L->z} for the zomplex case, and indices
in {\tt L->i}) are overwritten.

To compute an incremental factorization, select {\tt kstart} and {\tt kend} as the range
of rows of {\tt L} you wish to compute.    Rows {\tt kstart} to {\tt kend-1} of {\tt L}
will be computed.  A correct factorization will be computed
only if all descendants of all nodes {\tt kstart} to {\tt kend-1} in the elimination tree have
been factorized by a prior call to this routine, and if rows {\tt kstart} to {\tt kend-1}
have not been factorized.  This condition is {\bf not} checked on input.

In the symmetric case, {\tt A} must be stored in upper form ({\tt A->stype} is greater than zero).
The matrix {\tt F} is not accessed and may be {\tt NULL}.  Only
columns {\tt kstart} to {\tt kend-1} of {\tt A} are accessed.

In the unsymmetric case,
	the typical case is {\tt F=A'}.  Alternatively, if {\tt F=A(:,f)'}, then this
	routine factorizes the matrix {\tt S = beta*I + A(:,f)*A(:,f)'}.
	The product {\tt A*F} is assumed to be symmetric;
	only the upper triangular part of {\tt A*F} is used.
	{\tt F} must be of size {\tt A->ncol} by {\tt A->nrow}.

% J. Liu, "A compact row storage scheme for Cholesky factors", ACM Trans.
% Math. Software, vol 12, 1986, pp. 127-148.

%---------------------------------------
\subsection{{\tt cholmod\_rowfac\_mask}: row-oriented Cholesky factorization}
%---------------------------------------

\input{_rowfac_mask.tex}
For use in LPDASA only.


%---------------------------------------
\subsection{{\tt cholmod\_row\_subtree}: pattern of row of a factor}
%---------------------------------------

\input{_row_subtree.tex}
Compute the nonzero pattern of the solution to the lower triangular system
\begin{verbatim}
    L(0:k-1,0:k-1) * x = A (0:k-1,k)
\end{verbatim}
if {\tt A} is symmetric, or
\begin{verbatim}
    L(0:k-1,0:k-1) * x = A (0:k-1,:) * A (:,k)'
\end{verbatim}
if {\tt A} is unsymmetric.
This gives the nonzero pattern of row {\tt k} of {\tt L} (excluding the diagonal).
The pattern is returned postordered, according to the subtree of the elimination
tree rooted at node {\tt k}.

The symmetric case requires {\tt A} to be in symmetric-upper form.

The result is returned in {\tt R}, a pre-allocated sparse matrix of size {\tt nrow}-by-1,
with {\tt R->nzmax >= nrow}.  {\tt R} is assumed to be packed ({\tt Rnz [0]} is not updated);
the number of entries in {\tt R} is given by {\tt Rp [0]}.

%---------------------------------------
\subsection{{\tt cholmod\_row\_lsubtree}: pattern of row of a factor}
%---------------------------------------

\input{_row_lsubtree.tex}
Identical to {\tt cholmod\_row\_subtree}, except the elimination tree is
found from {\tt L} itself, not {\tt Parent}.  Also, {\tt F=A'} is not provided;
the nonzero pattern of the {\tt k}th column of {\tt F} is given by 
{\tt Fi} and {\tt fnz} instead.

%---------------------------------------
\subsection{{\tt cholmod\_resymbol}: re-do symbolic factorization}
%---------------------------------------

\input{_resymbol.tex}
Recompute the symbolic pattern of {\tt L}.  Entries not in the symbolic pattern
of the factorization of {\tt A(p,p)} or {\tt F*F'}, where
{\tt F=A(p,f)} or {\tt F=A(:,f)}, are dropped, where
{\tt p = L->Perm} is used to permute the input matrix {\tt A}.

Refer to {\tt cholmod\_transpose\_unsym} for a description of {\tt f}.

If an entry in {\tt L} is kept, its numerical value does not change.

This routine is used after a supernodal factorization is converted into
a simplicial one, to remove zero entries that were added due to relaxed
supernode amalgamation.  It can also be used after a series of downdates
to remove entries that would no longer be present if the matrix were
factorized from scratch.  A downdate ({\tt cholmod\_updown}) does not remove any
entries from {\tt L}.

%---------------------------------------
\subsection{{\tt cholmod\_resymbol\_noperm}: re-do symbolic factorization}
%---------------------------------------

\input{_resymbol_noperm.tex}
Identical to {\tt cholmod\_resymbol}, except that the fill-reducing
ordering {\tt L->Perm} is not used.

%---------------------------------------
\subsection{{\tt cholmod\_postorder}: tree postorder}
%---------------------------------------

\input{_postorder.tex}
Postorder a tree.  The tree is either an elimination tree (the output from
{\tt cholmod\_etree}) or a component tree (from {\tt cholmod\_nested\_dissection}).

An elimination tree is a complete tree of {\tt n} nodes with {\tt Parent [j] > j} or
{\tt Parent [j] = -1} if j is a root.  On output {\tt Post [0..n-1]} is a complete
permutation vector;
{\tt Post [k] = j} if node {\tt j} is the {\tt k}th node in the
postordered elimination tree, where {\tt k} is in the range 0 to {\tt n-1}.

A component tree is a subset of {\tt 0:n-1}.  {\tt Parent [j] = -2} if node {\tt j} is not
in the component tree.  {\tt Parent [j] = -1} if {\tt j} is a root of the component
tree, and {\tt Parent [j]} is in the range 0 to {\tt n-1} if {\tt j} is in the component
tree but not a root.  On output, {\tt Post [k]} is defined only for nodes in
the component tree.  {\tt Post [k] = j} if node {\tt j} is the {\tt k}th node in the
postordered component tree, where {\tt k} is in the range 0 to the number of
components minus 1.
Node {\tt j} is ignored and not included in the postorder if {\tt Parent [j] < -1}.
As a result, {\tt cholmod\_check\_parent (Parent, ...)} and {\tt cholmod\_check\_perm (Post, ...)}
fail if used for a component tree and its postordering.

An optional node weight can be given.  When starting a postorder at node {\tt j},
the children of {\tt j} are ordered in decreasing order of their weight.
If no weights are given ({\tt Weight} is {\tt NULL}) then children are ordered in
decreasing order of their node number.  The weight of a node must be in the
range 0 to {\tt n-1}.  Weights outside that range are silently converted to that
range (weights $<$ 0 are treated as zero, and weights $\ge$ {\tt n} are treated as {\tt n-1}).

%---------------------------------------
\subsection{{\tt cholmod\_rcond}: reciprocal condition number}
%---------------------------------------

\input{_rcond.tex}
Returns a rough estimate of the reciprocal of the condition number:
the minimum entry on the diagonal of {\tt L} (or absolute entry of {\tt D} for an $\m{LDL}\tr$
factorization) divided by the maximum entry.  {\tt L} can be real, complex, or
zomplex.  Returns -1 on error, 0 if the matrix is singular or has a zero or NaN
entry on the diagonal of {\tt L}, 1 if the matrix is 0-by-0, or
{\tt min(diag(L))/max(diag(L))} otherwise.  Never returns NaN; if {\tt L} has a NaN on
the diagonal it returns zero instead.

%-------------------------------------------------------------------------------
\newpage \section{{\tt Modify} Module routines}
%-------------------------------------------------------------------------------

%---------------------------------------
\subsection{{\tt cholmod\_updown}: update/downdate}
%---------------------------------------

\input{_updown.tex}
Updates/downdates the $\m{LDL}\tr$ factorization (symbolic, then numeric), by
computing a new factorization of
\[
\new{\m{L}}\new{\m{D}}\new{\m{L}}\tr = \m{LDL}\tr \pm \m{CC}\tr
\]
where $\new{\m{L}}$ denotes the new factor.
{\tt C} must be sorted.  It can be either packed or unpacked.  As in all CHOLMOD
routines, the columns of {\tt L} are sorted on input, and also on output.
If {\tt L} does not contain a simplicial numeric $\m{LDL}\tr$ factorization, it
is converted into one.  Thus, a supernodal $\m{LL}\tr$ factorization
can be passed to {\tt cholmod\_updown}.
A symbolic {\tt L} is converted into a numeric identity matrix.
If the initial conversion fails, the factor is returned unchanged.

If memory runs out during the update, the factor is returned as a simplicial
symbolic factor.  That is, everything is freed except for the fill-reducing
ordering and its corresponding column counts (typically computed by
{\tt cholmod\_analyze}).

Note that the fill-reducing permutation {\tt L->Perm} is not used.  The row
indices of {\tt C} refer to the rows of {\tt L}, not {\tt A}.  If your original system is
$\m{LDL}\tr = \m{PAP}\tr$ (where $\m{P} =$ {\tt L->Perm}), and you want to compute the $\m{LDL}\tr$
factorization of $\m{A}+\m{CC}\tr$, then you must permute $\m{C}$ first.  That is,
if
\[
    \m{PAP}\tr = \m{LDL}\tr
\]
is the initial factorization, then
\[
\m{P}(\m{A}+\m{CC}\tr)\m{P}\tr
    = \m{PAP}\tr+\m{PCC}\tr\m{P}\tr
    = \m{LDL}\tr + (\m{PC})(\m{PC})\tr
    = \m{LDL}\tr + \new{\m{C}}\new{\m{C}}\tr
\]
where $\new{\m{C}} = \m{PC}$.

You can use the {\tt cholmod\_submatrix} routine in the {\tt MatrixOps} Module
to permute {\tt C}, with:
\begin{verbatim}
    Cnew = cholmod_submatrix (C, L->Perm, L->n, NULL, -1, TRUE, TRUE, Common) ;
\end{verbatim}

Note that the {\tt sorted} input parameter to {\tt cholmod\_submatrix} must be {\tt TRUE},
because {\tt cholmod\_updown} requires {\tt C} with sorted columns.
Only real matrices are supported.
The algorithms are described in \cite{DavisHager99,DavisHager01}.

%---------------------------------------
\subsection{{\tt cholmod\_updown\_solve}: update/downdate}
%---------------------------------------

\input{_updown_solve.tex}
Identical to {\tt cholmod\_updown}, except
the system $\m{Lx}=\m{b}$ is also updated/downdated.
The new system is $\new{\m{L}}\new{\m{x}}=\m{b} + \Delta \m{b}$.  The old solution $\m{x}$ is overwritten
with $\new{\m{x}}$.  Note that as in the update/downdate of $\m{L}$ itself, the fill-
reducing permutation {\tt L->Perm} is not used.  The vectors $\m{x}$ and $\m{b}$ are in the permuted
ordering, not your original ordering.  This routine does not handle multiple right-hand-sides.

%---------------------------------------
\subsection{{\tt cholmod\_updown\_mark}: update/downdate}
%---------------------------------------

\input{_updown_mark.tex}
Identical to {\tt cholmod\_updown\_solve}, except that only part of $\m{L}$
is used in the update of the solution to $\m{Lx}=\m{b}$.  For more details,
see the source code file {\tt CHOLMOD/Modify/cholmod\_updown.c}.
This routine is meant for use in the {\tt LPDASA} linear program solver only,
by Hager and Davis.

%---------------------------------------
\subsection{{\tt cholmod\_updown\_mask}: update/downdate}
%---------------------------------------

\input{_updown_mask.tex}
For use in LPDASA only.

%---------------------------------------
\subsection{{\tt cholmod\_rowadd}: add row to factor}
%---------------------------------------

\input{_rowadd.tex}
Adds a row and column to an $\m{LDL}\tr$ factorization.  The {\tt k}th
row and column of {\tt L} must be equal to the {\tt k}th row and
column of the identity matrix on input.
Only real matrices are supported.
The algorithm is described in \cite{DavisHager05}.

%---------------------------------------
\subsection{{\tt cholmod\_rowadd\_solve}: add row to factor}
%---------------------------------------

\input{_rowadd_solve.tex}
Identical to {\tt cholmod\_rowadd}, except the system $\m{Lx}=\m{b}$ is also updated/downdated, just like {\tt cholmod\_updown\_solve}.

%---------------------------------------
\subsection{{\tt cholmod\_rowdel}: delete row from factor}
%---------------------------------------

\input{_rowdel.tex}
Deletes a row and column from an $\m{LDL}\tr$ factorization.  The {\tt k}th
row and column of {\tt L} is equal to the {\tt k}th row and
column of the identity matrix on output.
Only real matrices are supported.

%---------------------------------------
\subsection{{\tt cholmod\_rowdel\_solve}: delete row from factor}
%---------------------------------------

\input{_rowdel_solve.tex}
Identical to {\tt cholmod\_rowdel}, except the system $\m{Lx}=\m{b}$ is also updated/downdated, just like {\tt cholmod\_updown\_solve}.
When row/column $k$ of $\m{A}$ is deleted from the system $\m{Ay}=\m{b}$, this can induce
a change to $\m{x}$, in addition to changes arising when $\m{L}$ and $\m{b}$ are modified.
If this is the case, the kth entry of $\m{y}$ is required as input ({\tt yk}).
The algorithm is described in \cite{DavisHager05}.

%---------------------------------------
\subsection{{\tt cholmod\_rowadd\_mark}: add row to factor}
%---------------------------------------

\input{_rowadd_mark.tex}
Identical to {\tt cholmod\_rowadd\_solve}, except that only part of $\m{L}$
is used in the update of the solution to $\m{Lx}=\m{b}$.  For more details,
see the source code file {\tt CHOLMOD/Modify/cholmod\_rowadd.c}.
This routine is meant for use in the {\tt LPDASA} linear program solver only.

%---------------------------------------
\subsection{{\tt cholmod\_rowdel\_mark}: delete row from factor}
%---------------------------------------

\input{_rowdel_mark.tex}
Identical to {\tt cholmod\_rowadd\_solve}, except that only part of $\m{L}$
is used in the update of the solution to $\m{Lx}=\m{b}$.  For more details,
see the source code file {\tt CHOLMOD/Modify/cholmod\_rowdel.c}.
This routine is meant for use in the {\tt LPDASA} linear program solver only.

%-------------------------------------------------------------------------------
\newpage \section{{\tt MatrixOps} Module routines}
%-------------------------------------------------------------------------------

%---------------------------------------
\subsection{{\tt cholmod\_drop}: drop small entries}
%---------------------------------------

\input{_drop.tex}
Drop small entries from {\tt A}, and entries in the ignored part of {\tt A} if {\tt A}
is symmetric.  No CHOLMOD routine drops small numerical entries
from a matrix, except for this one.  NaN's and Inf's are kept.

Supports pattern and real matrices; complex and zomplex matrices are not supported.

%---------------------------------------
\subsection{{\tt cholmod\_norm\_dense}: dense matrix norm}
%---------------------------------------

\input{_norm_dense.tex}
Returns the infinity-norm, 1-norm, or 2-norm of a dense matrix.
Can compute the 2-norm only for a dense column vector.
All xtypes are supported.

%---------------------------------------
\subsection{{\tt cholmod\_norm\_sparse}: sparse matrix norm}
%---------------------------------------

\input{_norm_sparse.tex}
Returns the infinity-norm or 1-norm of a sparse matrix.
All xtypes are supported.

%---------------------------------------
\newpage \subsection{{\tt cholmod\_scale}: scale sparse matrix}
%---------------------------------------

\input{_scale.tex}
Scales a matrix:  {\tt A = diag(s)*A}, {\tt A*diag(s)}, {\tt s*A}, or {\tt diag(s)*A*diag(s)}.

{\tt A} can be of any type (packed/unpacked, upper/lower/unsymmetric).
The symmetry of {\tt A} is ignored; all entries in the matrix are modified.

If {\tt A} is {\tt m}-by-{\tt n} unsymmetric but scaled symmetrically, the result is
\begin{verbatim}
    A = diag (s (1:m)) * A * diag (s (1:n))
\end{verbatim}

Row or column scaling of a symmetric matrix still results in a symmetric
matrix, since entries are still ignored by other routines.
For example, when row-scaling a symmetric matrix where just the upper
triangular part is stored (and lower triangular entries ignored)
{\tt A = diag(s)*triu(A)} is performed, where the result {\tt A} is also
symmetric-upper.  This has the effect of modifying the implicit lower
triangular part.  In MATLAB notation:
\begin{verbatim}
    U = diag(s)*triu(A) ;
    L = tril (U',-1)
    A = L + U ;
\end{verbatim}

The scale parameter determines the kind of scaling to perform and the size of {\tt S}:

\begin{tabular}{lll}
\hline
{\tt scale} & operation & size of {\tt S} \\
\hline
{\tt CHOLMOD\_SCALAR} & {\tt s[0]*A}		& 1 \\
{\tt CHOLMOD\_ROW}    & {\tt diag(s)*A}		& {\tt nrow}-by-1 or 1-by-{\tt nrow} \\
{\tt CHOLMOD\_COL}    & {\tt A*diag(s)}		& {\tt ncol}-by-1 or 1-by-{\tt ncol} \\
{\tt CHOLMOD\_SYM}    & {\tt diag(s)*A*diag(s)}	& {\tt max(nrow,ncol)}-by-1, or 1-by-{\tt max(nrow,ncol)} \\
\hline
\end{tabular}

Only real matrices are supported.

%---------------------------------------
\newpage \subsection{{\tt cholmod\_sdmult}: sparse-times-dense matrix}
%---------------------------------------

\input{_sdmult.tex}
Sparse matrix times dense matrix:
{\tt Y = alpha*(A*X) + beta*Y} or {\tt Y = alpha*(A'*X) + beta*Y},
where {\tt A} is sparse and {\tt X} and {\tt Y} are dense.
When using {\tt A},
{\tt X} has {\tt A->ncol} rows and
{\tt Y} has {\tt A->nrow} rows.
When using {\tt A'},
{\tt X} has {\tt A->nrow} rows and
{\tt Y} has {\tt A->ncol} rows.
If {\tt transpose = 0}, then {\tt A} is used;
otherwise, {\tt A'} is used (the complex conjugate transpose).
The {\tt transpose} parameter is ignored if the matrix is symmetric or Hermitian.
(the array transpose {\tt A.'} is not supported).
Supports real, complex, and zomplex matrices, but the xtypes of {\tt A}, {\tt X}, and {\tt Y} must all match.

%---------------------------------------
\subsection{{\tt cholmod\_ssmult}: sparse-times-sparse matrix}
%---------------------------------------

\input{_ssmult.tex}
Computes {\tt C = A*B}; multiplying two sparse matrices.
{\tt C} is returned as packed, and either unsorted or sorted, depending on the
{\tt sorted} input parameter.  If {\tt C} is returned sorted, then either {\tt C = (B'*A')'}
or {\tt C = (A*B)''} is computed, depending on the number of nonzeros in {\tt A}, {\tt B}, and {\tt C}.
The stype of {\tt C} is determined by the {\tt stype} parameter.
Only pattern and real matrices are supported.  Complex and zomplex matrices
are supported only when the numerical values are not computed ({\tt values}
is {\tt FALSE}).

%---------------------------------------
\newpage \subsection{{\tt cholmod\_submatrix}: sparse submatrix}
%---------------------------------------

\input{_submatrix.tex}
Returns {\tt C = A (rset,cset)}, where {\tt C} becomes {\tt length(rset)}-by-{\tt length(cset)} in dimension.
{\tt rset} and {\tt cset} can have duplicate entries.  {\tt A} must be unsymmetric.  {\tt C} unsymmetric and
is packed.  If {\tt sorted} is {\tt TRUE} on input, or {\tt rset} is sorted and {\tt A} is
sorted, then {\tt C} is sorted; otherwise {\tt C} is unsorted.

If {\tt rset} is {\tt NULL}, it means ``{\tt [ ]}'' in MATLAB notation, the empty set.
The number of rows in the result {\tt C} will be zero if {\tt rset} is {\tt NULL}.
Likewise if {\tt cset} means the empty set; the number of columns in the result {\tt C} will be zero if {\tt cset} is {\tt NULL}.
If {\tt rsize} or {\tt csize} is negative, it denotes ``{\tt :}'' in MATLAB notation.
Thus, if both {\tt rsize} and {\tt csize} are negative {\tt C = A(:,:) = A} is returned.

For permuting a matrix, this routine is an alternative to {\tt cholmod\_ptranspose}
(which permutes and transposes a matrix and can work on symmetric matrices).

The time taken by this routine is O({\tt A->nrow}) if the {\tt Common} workspace needs
to be initialized, plus O({\tt C->nrow + C->ncol + nnz (A (:,cset))}).  Thus, if {\tt C}
is small and the workspace is not initialized, the time can be dominated by
the call to {\tt cholmod\_allocate\_work}.  However, once the workspace is
allocated, subsequent calls take less time.

Only pattern and real matrices are supported.  Complex and zomplex matrices
are supported only when {\tt values} is {\tt FALSE}.

%---------------------------------------
\newpage \subsection{{\tt cholmod\_horzcat}: horizontal concatenation}
%---------------------------------------

\input{_horzcat.tex}
Horizontal concatenation, returns {\tt C = [A,B]} in MATLAB notation.
{\tt A} and {\tt B} can have any stype.  {\tt C} is returned unsymmetric and packed.
{\tt A} and {\tt B} must have the same number of rows.
{\tt C} is sorted if both {\tt A} and {\tt B} are sorted.
{\tt A} and {\tt B} must have the same numeric xtype, unless {\tt values} is {\tt FALSE}.
{\tt A} and {\tt B} cannot be complex or zomplex, unless {\tt values} is {\tt FALSE}.

%---------------------------------------
\subsection{{\tt cholmod\_vertcat}: vertical concatenation}
%---------------------------------------

\input{_vertcat.tex}
Vertical concatenation, returns {\tt C = [A;B]} in MATLAB notation.
{\tt A} and {\tt B} can have any stype.  {\tt C} is returned unsymmetric and packed.
{\tt A} and {\tt B} must have the same number of columns.
{\tt C} is sorted if both {\tt A} and {\tt B} are sorted.
{\tt A} and {\tt B} must have the same numeric xtype, unless {\tt values} is {\tt FALSE}.
{\tt A} and {\tt B} cannot be complex or zomplex, unless {\tt values} is {\tt FALSE}.

%---------------------------------------
\newpage \subsection{{\tt cholmod\_symmetry}: compute the symmetry of a matrix}
%---------------------------------------

\input{_symmetry.tex}

Determines if a sparse matrix is rectangular, unsymmetric, symmetric,
skew-symmetric, or Hermitian.  It does so by looking at its numerical values
of both upper and lower triangular parts of a CHOLMOD "unsymmetric"
matrix, where A->stype == 0.  The transpose of A is NOT constructed.

If not unsymmetric, it also determines if the matrix has a diagonal whose
entries are all real and positive (and thus a candidate for sparse Cholesky
if A->stype is changed to a nonzero value).

Note that a Matrix Market "general" matrix is either rectangular or
unsymmetric.

The row indices in the column of each matrix MUST be sorted for this function
to work properly (A->sorted must be TRUE).  This routine returns EMPTY if
A->stype is not zero, or if A->sorted is FALSE.  The exception to this rule
is if A is rectangular.

If option == 0, then this routine returns immediately when it finds a
non-positive diagonal entry (or one with nonzero imaginary part).   If the
matrix is not a candidate for sparse Cholesky, it returns the value
{\tt CHOLMOD\_MM\_UNSYMMETRIC}, even if the matrix might in fact be symmetric or
Hermitian.

This routine is useful inside the MATLAB backslash, which must look at an
arbitrary matrix (A->stype == 0) and determine if it is a candidate for
sparse Cholesky.  In that case, option should be 0.

This routine is also useful when writing a MATLAB matrix to a file in
Rutherford/Boeing or Matrix Market format.  Those formats require a
determination as to the symmetry of the matrix, and thus this routine should
not return upon encountering the first non-positive diagonal.  In this case,
option should be 1.

If option is 2, this function can be used to compute the numerical and
pattern symmetry, where 0 is a completely unsymmetric matrix, and 1 is a
perfectly symmetric matrix.  This option is used when computing the following
statistics for the matrices in the UF Sparse Matrix Collection.

	numerical symmetry: number of matched off-diagonal nonzeros over
	the total number of off-diagonal entries.  A real entry $a_{ij}$,  $i \ne j$,
	is matched if $a_{ji} = a_{ij}$, but this is only counted if both
	$a_{ji}$ and $a_{ij}$ are nonzero.  This does not depend on {\tt Z}.
	(If A is complex, then the above test is modified; $a_{ij}$ is matched
	if $\mbox{conj}(a_{ji}) = a_{ij}$.

	Then numeric symmetry = xmatched / nzoffdiag, or 1 if nzoffdiag = 0.
 
	pattern symmetry: number of matched offdiagonal entries over the
	total number of offdiagonal entries.  An entry $a_{ij}$, $i \ne j$, is
	matched if $a_{ji}$ is also an entry.

	Then pattern symmetry = pmatched / nzoffdiag, or 1 if nzoffdiag = 0.
 
The symmetry of a matrix with no offdiagonal entries is equal to 1.

A workspace of size ncol integers is allocated; EMPTY is returned if this
allocation fails.

Summary of return values:

\begin{tabular}{ll}
{\tt EMPTY (-1)}		    & out of memory, stype not zero, A not sorted \\
{\tt CHOLMOD\_MM\_RECTANGULAR 1}    & A is rectangular \\
{\tt CHOLMOD\_MM\_UNSYMMETRIC 2}    & A is unsymmetric \\
{\tt CHOLMOD\_MM\_SYMMETRIC 3}	    & A is symmetric, but with non-pos. diagonal \\
{\tt CHOLMOD\_MM\_HERMITIAN 4}	    & A is Hermitian, but with non-pos. diagonal \\
{\tt CHOLMOD\_MM\_SKEW\_SYMMETRIC 5}    & A is skew symmetric \\
{\tt CHOLMOD\_MM\_SYMMETRIC\_POSDIAG 6} & A is symmetric with positive diagonal \\
{\tt CHOLMOD\_MM\_HERMITIAN\_POSDIAG 7} & A is Hermitian with positive diagonal \\
\end{tabular}

See also the {\tt spsym} mexFunction, which is a MATLAB interface for this code.

If the matrix is a candidate for sparse Cholesky, it will return a result
\newline
{\tt CHOLMOD\_MM\_SYMMETRIC\_POSDIAG} if real, or {\tt CHOLMOD\_MM\_HERMITIAN\_POSDIAG} if
complex.  Otherwise, it will return a value less than this.  This is true
regardless of the value of the option parameter.



%-------------------------------------------------------------------------------
\newpage \section{{\tt Supernodal} Module routines}
%-------------------------------------------------------------------------------

%---------------------------------------
\subsection{{\tt cholmod\_super\_symbolic}: supernodal symbolic factorization}
%---------------------------------------

\input{_super_symbolic.tex}
Supernodal symbolic analysis of the $\m{LL}\tr$ factorization of {\tt A}, {\tt A*A'}, or {\tt A(:,f)*A(:,f)'}.
This routine must be preceded by a simplicial symbolic analysis
({\tt cholmod\_rowcolcounts}).  See {\tt Cholesky/cholmod\_analyze.c} for an example of how to use
this routine.
The user need not call this directly; {\tt cholmod\_analyze} is a ``simple'' wrapper for this routine.
{\tt A} can be symmetric (upper), or unsymmetric.  The symmetric/lower form is not supported.
In the unsymmetric case {\tt F} is the normally transpose of {\tt A}.
Alternatively, if {\tt F=A(:,f)'} then {\tt F*F'} is analyzed.
Requires {\tt Parent} and {\tt L->ColCount} to be defined on input; these are the
simplicial {\tt Parent} and {\tt ColCount} arrays as computed by {\tt cholmod\_rowcolcounts}.
Does not use {\tt L->Perm}; the input matrices {\tt A} and {\tt F} must already be properly
permuted.  Allocates and computes the supernodal pattern of {\tt L}
({\tt L->super}, {\tt L->pi}, {\tt L->px}, and {\tt L->s}).
Does not allocate the real part ({\tt L->x}).

%---------------------------------------
\newpage \subsection{{\tt cholmod\_super\_numeric}: supernodal numeric factorization}
%---------------------------------------

\input{_super_numeric.tex}
Computes the numerical Cholesky factorization of {\tt A+beta*I} or {\tt A*F+beta*I}.  Only the
lower triangular part of {\tt A+beta*I} or {\tt A*F+beta*I} is accessed.  The
matrices {\tt A} and {\tt F} must already be permuted according to the fill-reduction
permutation {\tt L->Perm}.  {\tt cholmod\_factorize} is an "easy" wrapper for this code
which applies that permutation.
The input scalar {\tt beta} is real; only the real part ({\tt beta[0]} is used.

Symmetric case: {\tt A} is a symmetric (lower) matrix.  {\tt F} is not accessed and may be {\tt NULL}.
With a fill-reducing permutation, {\tt A(p,p)} should be passed for {\tt A}, where is
{\tt p} is {\tt L->Perm}.

Unsymmetric case: {\tt A} is unsymmetric, and {\tt F} must be present.  Normally, {\tt F=A'}.
With a fill-reducing permutation, {\tt A(p,f)} and {\tt A(p,f)'} should be passed as the
parameters {\tt A} and {\tt F}, respectively, where {\tt f} is a list of the subset of the columns of {\tt A}.

The input factorization {\tt L} must be supernodal ({\tt L->is\_super} is {\tt TRUE}).  It can
either be symbolic or numeric.  In the first case, {\tt L} has been analyzed by
{\tt cholmod\_analyze} or {\tt cholmod\_super\_symbolic}, but the matrix has not yet been
numerically factorized.  The numerical values are allocated here and the
factorization is computed.  In the second case, a prior matrix has been
analyzed and numerically factorized, and a new matrix is being factorized.
The numerical values of {\tt L} are replaced with the new numerical factorization.

{\tt L->is\_ll} is ignored on input, and set to {\tt TRUE} on output.  This routine always computes an $\m{LL}\tr$
factorization.  Supernodal $\m{LDL}\tr$ factorization is not supported.

If the matrix is not positive definite the routine returns {\tt TRUE}, but sets
{\tt Common->status} to {\tt CHOLMOD\_NOT\_POSDEF} and {\tt L->minor} is set to the column at
which the failure occurred.  Columns {\tt L->minor} to {\tt L->n-1} are set to zero.

If {\tt L} is supernodal symbolic on input, it is converted to a supernodal numeric
factor on output, with an xtype of real if {\tt A} is real, or complex if {\tt A} is
complex or zomplex.  If {\tt L} is supernodal numeric on input, its xtype must
match {\tt A} (except that {\tt L} can be complex and {\tt A} zomplex).  The xtype of {\tt A} and {\tt F}
must match.

%---------------------------------------
\newpage \subsection{{\tt cholmod\_super\_lsolve}: supernodal forward solve}
%---------------------------------------

\input{_super_lsolve.tex}
Solve $\m{Lx}=\m{b}$ for a supernodal factorization.  This routine does not
apply the permutation {\tt L->Perm}.  See {\tt cholmod\_solve} for a more general
interface that performs that operation.
Only real and complex xtypes are supported.
{\tt L}, {\tt X}, and {\tt E} must have the same xtype.

%---------------------------------------
\subsection{{\tt cholmod\_super\_ltsolve}: supernodal backsolve}
%---------------------------------------

\input{_super_ltsolve.tex}
Solve $\m{L}\tr\m{x}=\m{b}$ for a supernodal factorization.  This routine does not
apply the permutation {\tt L->Perm}.  See {\tt cholmod\_solve} for a more general
interface that performs that operation.
Only real and complex xtypes are supported.
{\tt L}, {\tt X}, and {\tt E} must have the same xtype.

%-------------------------------------------------------------------------------
\newpage \section{{\tt Partition} Module routines}
%-------------------------------------------------------------------------------

%---------------------------------------
\subsection{{\tt cholmod\_nested\_dissection}: nested dissection ordering}
%---------------------------------------

\input{_nested_dissection.tex}
CHOLMOD's nested dissection algorithm:
	using its own compression and connected-components
	algorithms, an external graph partitioner (METIS), and a constrained
	minimum degree ordering algorithm (CAMD, CCOLAMD, or CSYMAMD).  Typically
	gives better orderings than {\tt METIS\_NodeND} (about 5\% to 10\% fewer
	nonzeros in {\tt L}).

This method uses a node bisector, applied recursively (but using a
non-recursive implementation).  Once the graph is partitioned, it calls a
constrained minimum degree code (CAMD or CSYMAMD for {\tt A+A'},
and CCOLAMD for {\tt A*A'}) to
order all the nodes in the graph - but obeying the constraints determined
by the separators.  This routine is similar to {\tt METIS\_NodeND}, except for
how
it treats the leaf nodes.  {\tt METIS\_NodeND} orders the leaves of the separator
tree with {\tt MMD}, ignoring the rest of the matrix when ordering a single leaf.
This routine orders the whole matrix with CAMD, CSYMAMD, or CCOLAMD, all at once,
when the graph partitioning is done.

%---------------------------------------
\newpage \subsection{{\tt cholmod\_metis}: interface to METIS nested dissection}
%---------------------------------------

\input{_metis.tex}
CHOLMOD wrapper for the {\tt METIS\_NodeND} ordering routine.  Creates {\tt A+A'},
{\tt A*A'} or {\tt A(:,f)*A(:,f)'} and then calls {\tt METIS\_NodeND} on the resulting graph.
This routine is comparable to {\tt cholmod\_nested\_dissection}, except that it
calls {\tt METIS\_NodeND} directly, and it does not return the separator tree.

%---------------------------------------
\newpage \subsection{{\tt cholmod\_camd}: interface to CAMD}
%---------------------------------------

\input{_camd.tex}
CHOLMOD interface to the CAMD ordering routine.  Finds a permutation
{\tt p} such that the Cholesky factorization of {\tt A(p,p)}
is sparser than {\tt A}.  If {\tt A} is unsymmetric,
{\tt A*A'} is ordered.
If {\tt Cmember[i]=c} then node {\tt i} is in set {\tt c}.
All nodes in set 0 are ordered first, followed by all nodes in set 1, and so on.

%---------------------------------------
\newpage \subsection{{\tt cholmod\_ccolamd}: interface to CCOLAMD}
%---------------------------------------

\input{_ccolamd.tex}
CHOLMOD interface to the CCOLAMD ordering routine.  Finds a permutation
{\tt p} such that the Cholesky factorization of {\tt A(p,:)*A(p,:)'} is sparser than {\tt A*A'}.
The column elimination is found and postordered, and the CCOLAMD ordering is then
combined with its postordering.  {\tt A} must be unsymmetric.
If {\tt Cmember[i]=c} then node {\tt i} is in set {\tt c}.
All nodes in set 0 are ordered first, followed by all nodes in set 1, and so on.

%---------------------------------------
\subsection{{\tt cholmod\_csymamd}: interface to CSYMAMD}
%---------------------------------------

\input{_csymamd.tex}
CHOLMOD interface to the CSYMAMD ordering routine.  Finds a permutation
{\tt p} such that the Cholesky factorization of {\tt A(p,p)} is sparser than {\tt A}.
The elimination tree is found and postordered, and the CSYMAMD
ordering is then combined with its postordering.  If {\tt A} is unsymmetric,
{\tt A+A'} is ordered ({\tt A} must be square).
If {\tt Cmember[i]=c} then node {\tt i} is in set {\tt c}.
All nodes in set 0 are ordered first, followed by all nodes in set 1, and so on.

%---------------------------------------
\newpage
\subsection{{\tt cholmod\_bisect}: graph bisector}
%---------------------------------------

\input{_bisect.tex}
Finds a node bisector of {\tt A}, {\tt A*A'}, {\tt A(:,f)*A(:,f)'}:
a set of nodes that partitions the graph into two parts.
Compresses the graph first, ensures the graph is symmetric with
no diagonal entries, and then calls METIS.

%---------------------------------------
\subsection{{\tt cholmod\_metis\_bisector}: interface to METIS node bisector}
%---------------------------------------

\input{_metis_bisector.tex}
Finds a set of nodes that bisects the graph of {\tt A} or {\tt A*A'} (a direct interface to \newline
{\tt METIS\_NodeComputeSeparator}).

The input matrix {\tt A} must be square, symmetric (with both upper and lower
parts present) and with no diagonal entries.  These conditions are not
checked.  Use cholmod\_bisect to check these conditions.

%---------------------------------------
\newpage
\subsection{{\tt cholmod\_collapse\_septree}: prune a separator tree}
%---------------------------------------

\input{_collapse_septree.tex}
Prunes a separator tree obtained from {\tt cholmod\_nested\_dissection}.

\newpage
\bibliographystyle{plain}
\bibliography{UserGuide}
\end{document}
